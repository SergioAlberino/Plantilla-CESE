% Chapter 1

\chapter{Introducción general} % Main chapter title

\label{Chapter1} % For referencing the chapter elsewhere, use \ref{Chapter1} 
\label{IntroGeneral}
En este capítulo se presentan las características de los robots de servicio, se  reseña el uso de luz ultravioleta como germicida y se exponen los objetivos que motivaron el presente trabajo y su respectivo alcance.
%----------------------------------------------------------------------------------------

% Define some commands to keep the formatting separated from the content 
\newcommand{\keyword}[1]{\textbf{#1}}
\newcommand{\tabhead}[1]{\textbf{#1}}
\newcommand{\code}[1]{\texttt{#1}}
\newcommand{\file}[1]{\texttt{\bfseries#1}}
\newcommand{\option}[1]{\texttt{\itshape#1}}
\newcommand{\grados}{$^{\circ}$}

%----------------------------------------------------------------------------------------

%\section{Introducción}

%----------------------------------------------------------------------------------------
\section{Robots de servicio}

A lo largo del siglo XX la robótica pasó de ser una temática de la rama de la ciencia ficción, a cumplir un importante rol dentro de los complejos industriales. En los últimos años los robots han pasado a tener cada vez más tareas de “servicio” para ambientes  públicos y hogareños \citep{services}.
La robótica de servicios abarca un amplio campo de aplicaciones, la mayoría de las cuales tienen diferentes grados de automatización, desde la teleoperación completa hasta el funcionamiento autónomo, y constituye un campo de aplicación más diverso que el de la robótica industrial. En la  figura \ref{fig:robotsservicio} se pueden observar tres tipos de robots de servicios: una aspiradora hogareña, un cortador de césped y un limpiavidrios.

\begin{figure}[h]
	\centering
	\includegraphics[width=\textwidth]{./Figures/robotsservicio.jpg}
	\caption{Robots de servicio\protect\footnotemark.}
	\label{fig:robotsservicio}
\end{figure}
\footnotetext{Imágenes tomadas de \url{https://www.domotizar.com/}}


A mediados de la década de 1990, la Comisión Económica de las Naciones Unidas para Europa (UNECE) \citep{UNECE} y la Federación Internacional de Robótica (IFR) \citep{IFR} adoptaron un sistema de clasificación de robots de servicio dividida por categorías y tipos de interacción, que se ha mantenido hasta la actualidad. En la  figura \ref{fig:clasificacion} se puede observar los primeros ítems de clasificación para robots domésticos/personales de acuerdo a los tipos y áreas de aplicación.
\pagebreak

\begin{figure}[h]
	\centering
	\includegraphics[width=\textwidth]{./Figures/clasificacion.png}
	\caption{Clasificación de robots de servicio\protect\footnotemark.}
	\label{fig:clasificacion}
\end{figure}
\footnotetext{Imagen tomada de \url{https://www.editores-srl.com.ar/sites/default/files/aa1_ifr_robots.pdf}}




\subsection{Robots móviles para inspección y limpieza}

Los robots móviles son dispositivos que poseen un sistema de locomoción capaz de navegar a través de un determinado ambiente de trabajo. Normalmente cuentan con un cierto nivel de autonomía que les permite el desplazamiento sin colisiones por un recorrido específico. Sus aplicaciones son muchas y en general  están relacionadas con tareas monótonas o riesgosas para la salud humana\citep{whatare}.

Las plataformas móviles pueden realizar tareas de inspección y limpieza de manera autónoma o controlada remotamente por un operador. Son utilizadas en zonas de difícil acceso debido a limitaciones de espacio o razones de seguridad. Este tipo de robot suele contar con sensores de distinto tipo para detectar los límites y obstáculos del entorno.
 
La proliferación de robots para limpieza se incrementó fuertemente a partir de la pandemia de Covid-19, con lo que se los puede encontrar hoy en día en espacios en los que antes no estaban presentes, tales como salas médicas,  hoteles y en el transporte público \citep{Cleaning}. 

Estos dispositivos “de interior” abarcan varios tipos. En la  figura \ref{fig:robotslimpieza} se puede observar un modelo de robot trapeador húmedo, una aspiradora robótica y un limpiavidrios automático, a modo de ejemplo.

\begin{figure}[h]
	\centering
	\includegraphics[width=\textwidth]{./Figures/robotslimpieza.jpg}
	\caption{Distintos tipos de robot de limpieza\protect\footnotemark.}
	\label{fig:robotslimpieza}
\end{figure}
\footnotetext{Imágenes tomadas de \url{https://www.domotizar.com/}}

Si bien todas las aspiradoras robots cumplen con la misma función básica de aspirar polvo y suciedad, las prestaciones de cada modelo y marca varían considerablemente junto con el precio de mercado. Los precios pueden oscilar entre los 200 y los 1200 dólares \citep{roomba}.
Los robots de mayor gama incorporan cepillos, trapeadores húmedos y/o luz ultravioleta germicida. Las versiones más avanzadas  presentan mayor cantidad de sensores de proximidad a la vez que incorporan cámaras y rayos láser para medir distancias hasta los obstáculos y planificar recorridos. 
La navegación de los robots más simples es de tipo aleatoria, simplemente  sorteando los obstáculos con los que se encuentra y sin seguir una trayectoria ordenada. 




%----------------------------------------------------------------------------------------
\section{Desinfección usando luz ultravioleta}

El espectro ultravioleta (UV) abarca la banda de radiación electromagnética entre los 400 y 100 nm, presentando una longitud de onda menor que la de la luz visible y mayor que la de los rayos X \citep{lit}. Se divide en las siguientes categorías principales:

\begin{itemize}
	\item los rayos UV-A (400 – 315 nm), que son los más cercanos al espectro visible.
	\item los rayos UV-B (315 – 280 nm), que son absorbidos en gran parte por diferentes elementos a medida que atraviesan el cielo.
	\item los rayos UV-C (280 – 200 nm), que son absorbidos totalmente por la capa de ozono.
\end{itemize}

En la  figura \ref{fig:espectro} se observa la  clasificación de luz según longitud de onda en  espectro de radiación electromagnética.

\begin{figure}[h]
	\centering
	\includegraphics[width=14cm]{./Figures/espectro.PNG}
	\caption{Clasificación de luz según longitud de onda\protect\footnotemark.}
	\label{fig:espectro}
\end{figure}
\footnotetext{Imagen tomada de \url{https://www.lit-uv.com/es/technology/}}

%\pagebreak

La utilización de luz ultravioleta UV-C como germicida ha demostrado efectividad para la esterilización  las bacterias, gérmenes, virus, algas y esporas. 

Los virus tienen un tamaño inferior a un micrómetro (µm, una millonésima parte de un metro) y las bacterias son típicamente de 0,5 a 5 µm. Técnicamente es incorrecto decir que los rayos  UV-C matan a los virus, siendo que no se trata de organismos vivientes. Sin embargo, el comité de foto-biología de la \emph{ Illuminating Engineering Society} (IES) informa que los fotones UV-C interactúan con el ARN y las moléculas de ADN en un virus o bacteria de modo que se evita su reproducción y por lo tanto su efecto infeccioso. Una célula que no puede reproducirse se considera muerta; ya que no puede multiplicarse dentro del anfitrión. A este proceso se lo denomina “desactivación”   \citep{IES}. En la figura \ref{fig:adn} se representa el efecto de la luz UV-C en el ADN de bacterias, gérmenes, virus, algas y esporas. 
 

\begin{figure}[h]
	\centering
	\includegraphics[width=9cm]{./Figures/adn.png}
	\caption{Efecto de la UV-C sobre el ADN de microorganismos\protect\footnotemark.}
	\label{fig:adn}
\end{figure}
\footnotetext{Imagen tomada de \url{https://www.lit-uv.com/es/technology/}}

La \emph{International Ultraviolet Association}  (IUVA) afirma que los resultados de pruebas en laboratorio de desinfección utilizando UV-C entre los 200 y 280 nm demuestran especial utilidad para reducir la transmisión de los virus causantes del COVID-19:  SARS-CoV-1 y MERS-CoV \citep{IUA}. En la práctica, el efecto depende de factores tales como  el tiempo de exposición y obstrucción que puedan tener los rayos para alcanzar plenamente los pliegues u ondulaciones que pudiera tener la superficie a desinfectar. 

En un informe sobre utilización de la radiación ultravioleta para desinfección \citep{CSIC}, el Consejo Superior de Investigaciones Científicas de España, concluye que el uso de radiación UV-C es muy adecuado para la desinfección de microorganismos y de virus, y propone su uso en combinación con métodos tradicionales para la desinfección en  zonas de alta contaminación.

La cantidad de inactivación de patógenos en superficies es directamente proporcional a la dosis de radiación UV-C, definida como el producto de la intensidad (W/m2) por la duración de la exposición. La dosis de luz UV necesaria varía en función del patógeno que se quiera desinfectar y de las condiciones ambientales\citep{CIE}.

Una de las ventajas de este método de desinfección radica en que una vez terminada el ambiente puede volver a ser utilizado inmediatamente, ya que no existe radiación persistente. La desinfección  por UV-C resulta además menos contaminante para el medio ambiente, al no exponer al ser humano a los riesgos derivados del uso de productos químicos. La desinfección germicida por ultravioleta es especialmente recomendada cuando debe realizarse sobre materiales que podrían verse afectados o dañados ante la limpieza continua con productos a base de químicos líquidos, como ser dispositivos electrónicos o materiales susceptibles a la  oxidación \citep{interior}. 
%En la figura \ref{fig:gabineteuv} se muestra un gabinete utilizado para la desinfección ultravioleta de toallas, pinzas, cortauñas, limpieza de cepillos de dientes, teléfono móvil, etc.
%%\vspace{12mm}
%
%\begin{figure}[h]
%	\centering
%	\includegraphics[width=7cm]{./Figures/gabineteuv.jpg}
%	\caption{Ejemplo de gabinete de desinfección ultravioleta.}
%	\label{fig:gabineteuv}
%\end{figure}



La desinfección por rayos UV-C es también útil en el caso de superficies de difícil acceso por su ubicación o cuando la zona presenta formas y estructuras que no permiten la higienización por contacto con paños o rociadores. 

Por otra parte, la irradiación con luz UV-C presenta una serie de limitaciones. El patógeno que desea esterilizarse debe ser  expuesto directamente a la radiación para que sea eliminado. Si bien la Organización Mundial de la Salud (OMS) recomienda el uso de rayos UV-C para desinfección, también alerta sobre los riesgos de  exposición en seres humanos y animales, cuya piel puede verse irritada, a la vez que puede producir daños a la vista \citep{MYTH}. En este sentido promueven la limpieza periódica de manos con jabón o con alcohol, y dejan la esterilización con UV-C para  instrumental y objetos de uso diario.





%----------------------------------------------------------------------------------------

\subsection{Robots de desinfección por luz ultravioleta}

Desde hace tiempo se utiliza luz ultravioleta para la desinfección de agua potable, y más recientemente ha sido incorporada como método germicida en conductos de ventilación. También se ha utilizado para la desinfección de instrumental e insumos en ambientes hospitalarios\citep{inventos}. 

En los últimos dos años, con el aumento de los cuidados de la salud debido a la pandemia mundial por el COVID-19, comenzaron a comercializarse robots móviles de luz ultravioleta germicida, para desinfectar quirófanos y salas de hospitales\citep{interempresas}.  
Estos robots, poseen paneles con tubos ultravioleta para poder irradiar completamente una habitación o parte de la misma y son de alturas entre los 120 y los 180 cm para poder iluminar camas y mesas desde arriba y poder pasar por puertas y aberturas convencionales. En la figura \ref{fig:robotuv} se muestra un robot de estas características.

\begin{figure}[h]
	\centering
	\includegraphics[width=12cm]{./Figures/robothospitalario.PNG}
	\caption{Robot desinfectante por luz ultravioleta en una sala de hospital\protect\footnotemark.}
	\label{fig:robotuv}
\end{figure}
\footnotetext{Imagen tomada de \url{https://www.infoplc.net/plus-plus/empresas/item/107726-mts-tech-robot-movil-ultravioleta-covid-19}}

La cobertura de los robots móviles de luz ultravioleta germicida suele ser mayor a los 180 grados, por lo que resulta importante no solo el recorrido realizado para abarcar todas las superficies, sino también evitar la presencia cercana de personas ya que la exposición de rayos ultravioleta puede ser perjudicial para la piel y la vista.  

Además de los hospitales, estos robots están siendo usados en otros espacios, como vagones del subterráneo, espacios comunes en hoteles, oficinas y áreas de control en aeropuertos \citep{masrobots}.


En 2020, la empresa argentina UV- Robotics lanzó el UVR-Robot \citep{UVR} que cuenta con tubos de luz UV-C germicida dispuestos en un arreglo de 360 grados para que la luz llegue hasta cualquier rincón y con una plataforma omnidireccional. El proceso toma  entre 5 y 15 minutos según la superficie, y se lo puede dirigir a control remoto. Esto le permite desinfectar autobuses, aviones y otros medios de transporte, salas de espera, centros de mayores, colegios, entidades bancarias, hoteles, ascensores o aseos. La iniciativa  tuvo el apoyo del Ministerio de Desarrollo Nacional y cuenta con validaciones y homologaciones de la Universidad Tecnológica Nacional. En la figura \ref{fig:uvrobot} se observa al UVR-Robot desinfectando un vagón de tren subterráneo. 

\begin{figure}[h]
	\centering
	\includegraphics[width=12cm]{./Figures/uvrobot.jpeg}
	\caption{UVR-Robot de la agencia nacional UV- Robotics\protect\footnotemark.}
	\label{fig:uvrobot}
\end{figure}
\footnotetext{Imagen tomada de \url{https://www.interempresas.net/Tecnologia-aulas/Articulos/321010-UVR-bot-reto-acabar-cualquier-rastro-covid-19-20-minutos-luz-ultravioleta.html}}

A nivel hogareño, muchos robots de limpieza empiezan a incorporar la luz ultravioleta como medio de desinfección. En general, es una característica adicional que presentan las aspiradoras robóticas que además de barrido y trapeado húmedo agregan la esterilización de suelos con luz ultravioleta, resultando hoy en día una característica evaluada en las comparativas de los distintos productos  \citep{Bidcom}. En la figura \ref{fig:coradiruv} se observa la imagen de una aspiradora modelo Warptech ARobot 1000 UV con la funcionalidad de desinfección por UV-C por medio de LEDs ultravioletas en la parte inferior.


\begin{figure}[h]
	\centering
	\includegraphics[width=5cm]{./Figures/coradiruv.jpg}
	\caption{Imagen de la aspiradora Warptech ARobot 1000 UV\protect\footnotemark.}
	\label{fig:coradiruv}
\end{figure}
\footnotetext{Imagen tomada de \url{https://www.coradir.com.ar/producto/productos_categoria/aspiradoras-roboticas}}


También se han comenzado a comercializar  robots móviles de las dimensiones de una aspiradora robot, pero con el único fin de esterilizar gérmenes y virus usando la emisión de luz UV-C de baja intensidad (sin aspiradora o trapeador), para empresas y comercios en general. La propuesta que presentan es la de dejar al robot funcionando por la noche en los espacios que se desean desinfectar. Un ejemplo de esta aplicación es el robot germicida móvil por UV-C y ozono  Conga Apolo \citep{conga} de la empresa española Cecotec. Su precio se encuentra alrededor de los 1000 euros.

Existen también dispositivos portátiles de esterilización con LEDs ultravioleta en la longitud de onda entre los 200 y 280 nm para desinfectar celulares, teclados, llaves, juguetes para bebés y otros objetos pequeños. Se trata de luz de baja potencia (menor a los 15 mW) pero fuertemente enfocada sobre la superficie a esterilizar, desde una distancia muy corta (menor a los 3 cm), con lo que aprovechan al máximo la energía radiada. El tiempo de  aplicación de la luz UV sobre la superficie oscila entre los 30 y 60 segundos, según la potencia del dispositivo. En la figura \ref{fig:esterilizador} se observa una imagen de un modelo de esterilizador portátil por UV-C. Muchos de estos dispositivos justifican su eficiencia con un estudio realizado en 2020 por el Guangdong Detection Center of Microbiology \citep{Guangdong}.   


\begin{figure}[h]
	\centering
	\includegraphics[width=4cm]{./Figures/esterilizador.PNG}
	\caption{Ejemplo de esterilizador portátil por UVC\protect\footnotemark.}
	\label{fig:esterilizador}
\end{figure}
\footnotetext{Imagen tomada de \url{https://procid.cl/producto/luz-led-germicida-portatil-uv-c-blanco/}}






%----------------------------------------------------------------------------------------

\section{Motivación}

Existen cada vez más robots de servicio orientados a tareas específicas de ayuda para la industria y para el hogar. 
Debido al avance de la pandemia por COVID-19, en 2020 proliferaron los robots  para desinfección utilizando emisión de luz ultravioleta de tipo germicida. Se trata de dispositivos de grandes dimensiones (con alturas entre uno y dos metros) ya que buscan cubrir superficies amplias como las de salas de hospitales, almacenes, espacios de transportes públicos, etc. Al mismo, tiempo se ha visto que muchos robots de limpieza hogareña empiezan a incorporar la desinfección por UV-C como parte de sus prestaciones. 
En un informe sobre utilización de la radiación ultravioleta para desinfección \citep{CSIC} elaborado por el Consejo Superior de Investigaciones Científicas de España, se propone como algo necesario el diseño de robots móviles que recorran superficies horizontales de forma autónoma irradiando luz UV-C para la desinfección de ambientes cerrados.

En función de estas cuestiones, y con el fin de elaborar un trabajo acorde al nivel de los temas planteados en la especialización en sistemas embebidos, es que surge la idea de construir una plataforma móvil de dimensiones similares a las de una aspiradora robot comercial, que pudiera utilizarse para desinfección por rayos UV-C en espacios cerrados y sobre superficies planas, donde puedan existir obstáculos (patas de mesas, sillas, etc.) que dificulten la limpieza por otros medios (trapeadores, cepillos). El dispositivo puede ser usado para desinfección sin residuos químicos en espacios públicos (salas de atención médica, instituciones educativas) y en el hogar. También podría resultar conveniente para esterilizar superficies que no pueden ser limpiadas con productos líquidos, como es el caso de losetas de caucho o productos similares, como las utilizadas en jardines de infantes o gimnasios. En la figura \ref{fig:ruvot} se observa una representación del robot para tareas de desinfección con UV-C.


\begin{figure}[h]
	\centering
	\includegraphics[width=10cm]{./Figures/rUVot14.png}
	\caption{Repressentación del robot para tareas de desinfección por UV-C.}
	\label{fig:ruvot}
\end{figure}


Si bien existen plataformas robot con fines educativos y de experimentación, la mayoría son de fabricación extranjera y de costos elevados para ser afrontados por instituciones educativas. La construcción de prototipos robóticos a nivel nacional constituye en ese sentido un buen aporte para ampliar el parque de plataformas de experimentación y desarrollo de nuevas aplicaciones.

Se espera que el hardware resultante del presente trabajo pueda ser aprovechado en el Grupo de Inteligencia Artificial y Robótica (GIAR) de la UTN - Facultad Regional Buenos Aires, para la evaluación de algoritmos de Inteligencia Artificial. La placa EDU-CIAA cuenta con capacidad suficiente para el procesamiento de algoritmos reactivos para la planificación de recorridos con obstáculos (compatibles con trabajos realizados), en los que intervenga aprendizaje por refuerzo o comportamientos “aprendidos” con una red neuronal. 

%----------------------------------------------------------------------------------------

\section{Objetivos y alcances}


El propósito de este trabajo es el desarrollo de un prototipo de robot móvil para ser utilizado en tareas de desinfección por efecto de rayos ultravioletas germicidas.
El dispositivo posee un modo autónomo en el que realiza un recorrido aleatorio dentro de una habitación evitando obstáculos detectados por los sensores, y un modo de teleoperación en el que puede controlarse a distancia (de unos metros) desde una aplicación en un celular o tablet a través de Bluetooth.


En todos los casos, el uso de este robot móvil constituye un aporte a la desinfección de ambientes cerrados, que no implica que deba dejarse de lado los métodos tradicionales de prevención y limpieza. 

%----------------------------------------------------------------------------------------
%	SECTION 2
%-----------------------------------------------------------------
\section{Requerimientos}

En esta sección se enumeran los requerimientos planteados en la planificación inicial del proyecto.  Los  requerimientos se han dividido en funcionales y no funcionales.

\label{sec:requerimientos}

\begin{enumerate}
\item Requerimientos funcionales
	\begin{enumerate}
	\item Capacidad de locomoción.  El robot debe ser capaz de desplazarse por medio de ruedas motorizadas, a través de superficies planas.
	\item Capacidad de percepción. El robot debe ser capaz de detectar y obtener información del medio. 
	\item Capacidad de comunicación inalámbrica. El robot sebe ser capaz de establecer una comunicación  por medio de un módulo Bluetooth, con una aplicación android en un celular o tablet.
	\item El robot deberá funcionar con alimentación a batería recargable.
	\item El proyecto debe ser extensible a una posible herramienta de enseñanza e investigación.

	\end{enumerate}
\item Requerimientos no funcionales
	\begin{enumerate}
	\item El robot no debe emitir luz UV-C cuando detecte movimiento a su alrededor, para no producir daños a la salud de personas o animales con los que interactue.
	\item El diseño del robot debe respetar regulaciones en cuanto a radiación en el espectro ultravioleta.
	\item Se utilizarán componentes electrónicos disponibles comercialmente en Argentina.
	\end{enumerate}
\end{enumerate}

%\section{Planificación}
%
%El trabajo se organizó para ser terminado en el mes de junio de 2021 con una dedicación aproximada de 600 horas en total, mientras se realizaba la cursada de la especialización en sistemas embebidos.
%
%\subsection{Diagrama de Gantt}
%Con el fin de organizar y dar seguimiento a las actividades requeridas y poder identificar los desvíos en los tiempos de ejecución programados, se cuantificaron los tiempos de las diversas tareas mediante el diagrama de Gantt, que se observa en las figuras \ref{fig:gantt1} y \ref{fig:gantt2}.
%
%
%\begin{figure}[htpb]
%\centering 
%\includegraphics[width=\textwidth]{./Figures/gantttabla.PNG}
%\caption{Tabla de tareas de Gantt.}
%\label{fig:gantt1}
%\end{figure}
%
%\begin{figure}[htpb]
%\centering 
%\includegraphics[width=\textwidth]{./Figures/Gantt.PNG}
%\caption{Diagrama de Gantt.}
%\label{fig:gantt2}
%\end{figure}
%
%\pagebreak
%
%\subsection{Diagrama de Precedencias}
%Se confeccionó también un diagrama de Precedencias o de Activity on Node (AON), con la finalidad de resaltar las tareas cuyos retrasos podrían resultar críticos para la concreción del trabajo. En rojo se indica el camino crítico, como puede apreciarse en la figura \ref{fig:AoN}
%
%\begin{figure}[htpb]
%\centering 
%\includegraphics[width=12cm]{./Figures/AoN.png}
%%\includegraphics[width=\textwidth]{./Figures/AoN.png}
%\caption{diagrama de Precedencias o de Activity on Node (AON).}
%\label{fig:AoN}
%\end{figure}
%\pagebreak

\subsubsection{Supuestos iniciales del proyecto}

Para el desarrollo de este trabajo se supuso inicialmente que:

\begin{itemize}
	\item Se iba a contar con disponibilidad de los laboratorios e instrumental de la  Secretaría de Ciencia, Tecnología e innovación productiva. UTN. Buenos Aires, para cubrir la tarea de desarrollo.
	\item Se iba a disponer de tiempo durante la jornada laboral para la realización del mismo. 
	\item Se iba a disponer de todos los componentes y herramientas necesarios.
\end{itemize}

Estos supuestos, incluidos en la planificación del trabajo, se cumplieron solo en parte, ya que la pandemia por COVID-2019 limitaron el acceso a los laboratorios e instrumental, a la vez que se incrementó el tiempo necesario para las actividades laborales desarrolladas en paralelo al proyecto. 






%----------------------------------------------------------------------------------------