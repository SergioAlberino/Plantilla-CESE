% Chapter 1

\chapter{Introducción general} % Main chapter title

\label{Chapter1} % For referencing the chapter elsewhere, use \ref{Chapter1} 
\label{IntroGeneral}
En este capítulo se presentan las características de los robots de servicio, se  reseña el uso de luz ultravioleta como germicida y se exponen los objetivos que motivaron el presente trabajo y sus respectivo alcance.
%----------------------------------------------------------------------------------------

% Define some commands to keep the formatting separated from the content 
\newcommand{\keyword}[1]{\textbf{#1}}
\newcommand{\tabhead}[1]{\textbf{#1}}
\newcommand{\code}[1]{\texttt{#1}}
\newcommand{\file}[1]{\texttt{\bfseries#1}}
\newcommand{\option}[1]{\texttt{\itshape#1}}
\newcommand{\grados}{$^{\circ}$}

%----------------------------------------------------------------------------------------

%\section{Introducción}

%----------------------------------------------------------------------------------------
\section{Robots de servicio}

A lo largo del siglo XX la robótica pasó de ser una temática de la rama de la ciencia ficción, a cumplir un importante rol dentro de los complejos industriales. En los últimos años los robots han pasado a tener cada vez más tareas de “servicio” para ambientes  públicos y hogareños.
La robótica de servicios abarca un amplio campo de aplicaciones, la mayoría de las cuales tienen diferentes grados de automatización, desde la teleoperación completa hasta la funcionamiento autónomo, y constituye un campo de aplicación más diverso que el de la robótica industrial. En la  figura \ref{fig:robotsservicio} se pueden observar tres tipos de robots de servicios: una aspiradora hogareña, un cortador de césped y un limpiavidrios.

\begin{figure}[h]
	\centering
	\includegraphics[width=\textwidth]{./Figures/robotsservicio.jpg}
	\caption{robots de servicio.\protect\footnotemark.}
	\label{fig:robotsservicio}
\end{figure}
\footnotetext{Imágenes tomadas de \url{https://www.domotizar.com/}}


A mediados de la década de 1990, la Comisión Económica de las Naciones Unidas para Europa (UNECE) .\citep{UNECE} y la Federación Internacional de Robótica (IFR) .\citep{IFR} adoptaron un sistema de clasificación de robots de servicio dividida por categorías y tipos de interacción, que se ha mantenido clasificación actual. En la  figura \ref{fig:robotsservicio} se puede observar los primeros ítems de clasificación para robots domésticos/personales de acuerdo a los tipos y áreas de aplicación.



\begin{figure}[h]
	\centering
	\includegraphics[width=\textwidth]{./Figures/clasificacion.png}
	\caption{clasificación de robots de servicio.\protect\footnotemark.}
	\label{fig:clasificacion}
\end{figure}
\footnotetext{Imagen tomada de \url{https://www.editores-srl.com.ar/sites/default/files/aa1_ifr_robots.pdf}}




\subsection{Robots móviles para inspección y limpieza}

Los robots móviles son dispositivos que poseen un sistema de locomoción capaz de navegar a través de un determinado ambiente de trabajo. Normalmente cuentan con cierto nivel de autonomía que les permite el desplazamiento sin colisiones por un recorrido específico. Sus aplicaciones son muchas y en general  están relacionadas con tareas monótonas o riesgosas para la salud humana.
Las plataformas móviles pueden realizar tareas de inspección y limpieza de manera autónoma o controlada remotamente por un operador. Son utilizadas en zonas de difícil acceso debido a limitaciones de espacio o razones de seguridad.  
Este tipo de robot suele contar con sensores de distinto tipo, para detectar los límites y obstáculos ante los que se presentan. 
La proliferación de de robots para limpieza se incrementó fuertemente a partir de la pandemia de Covid-19, con lo que se los puede encontrar hoy en día en espacios en los que antes no estaban presentes, tales como salas médicas,  hoteles y en el transporte público  .\citep{Cleaning}. Estos dispositivos “de interior” abarcan a la aspiradoras robóticas y a los robots de lavado de pisos que limpian pisos con funciones de barrido y trapeado húmedo. 


\subsubsection{Robots de limpieza UVC}

Acá va una comparativa de robots de limpieza UVC.
.
.
.
.
.


%----------------------------------------------------------------------------------------

\section{Desinfección usando Luz ultravioleta}

El espectro ultravioleta (UV) abarca la banda de radiación electromagnética entre los 400 y 100 nm, presentando una longitud de onda menor que la de la luz visible y mayor que la de los rayos X.  Se divide en tres las siguientes categorías principales: los rayos UV-A (400 – 315 nm), que son los más cercanos al espectro visible; los rayos UV-B (315 – 280 nm), que son absorbidos en gran parte por diferentes elementos a medida que atraviesan el cielo y los rayos UV-C (280 – 200 nm), que son absorbidos totalmente por la capa de ozono. En la  figura \ref{fig:espectro} se observa detalle de parte del espectro de radiación electromagnética y  su clasificación según longitud de onda.


\begin{figure}[h]
	\centering
	\includegraphics[width=\textwidth]{./Figures/espectro.PNG}
	\caption{clasificación según longitud de onda.\protect\footnotemark.}
	\label{fig:espectro}
\end{figure}
\footnotetext{Imagen tomada de \url{https://www.lit-uv.com/es/technology/}}

La utilización de luz ultravioleta UV-C como germicida ha demostrado efectividad para la esterilización  las bacterias, gérmenes, virus, algas y esporas. 

Los virus tienen un tamaño inferior a un micrómetro (µm, una millonésima parte de un metro) y las bacterias son típicamente de 0,5 a 5 µm. Técnicamente es incorrecto decir que los rayos  UV-C matan a los virus, siendo que no se trata de organismos vivientes. Sin embargo, el comité de foto-biología de la Illuminating Engineering Society (IES) informa que los fotones UV-C interactúan con el ARN y las moléculas de ADN en un virus o bacteria de modo que se evita su reproducción y por lo tanto su efecto infeccioso. A este proceso se lo denomina “desactivación”  .\citep{IES}.

La International Ultraviolet Association (IUVA) afirma que los resultados de pruebas en laboratorio de desinfección utilizando UV-C entre los 200 y 280nm demuestran especial utilidad para reducir la transmisión de los virus causantes del COVID-19:  SARS-CoV-1 y MERS-CoV .\citep{IUA}. En la práctica, el efecto depende de factores tales como  el tiempo de exposición y obstrucción que puedan tener los rayos para alcanzar plenamente los pliegues u ondulaciones que pudiera tener la superficie a desinfectar. 

Este tipo de desinfección, que no genera residuos químicos, es especialmente recomendada cuando debe realizarse sobre materiales que podrían verse afectados o dañados ante la limpieza continua con productos a base de alcoholes líquidos, como ser dispositivos electrónicos o materiales susceptibles de oxidación. También es especialmente aplicable en el caso de superficies de difícil acceso por su ubicación o por presentar formas y estructuras que no permiten la higienización por contacto con paños o rociadores. 
Por otra parte, si bien la Organización Mundial de la Salud (OMS) recomienda el uso de rayos UV-C para desinfección, también alerta sobre los riesgos de  exposición en seres humanos y animales, cuya piel puede verse irritada, a la vez que puede producir daños a la vista [6]. En este sentido promueven la limpieza de manos periódica con jabón o con alcohol, y dejan la esterilización con UV-C para  instrumental y objetos de uso diario.


%----------------------------------------------------------------------------------------

\section{Qué incluye esta plantilla}

\subsection{Carpetas}

Esta plantilla se distribuye como una único archivo .zip que se puede descomprimir en varios archivos y carpetas. Asimismo, se puede consultar el repositorio git para obtener la última versión de los archivos, \url{https://github.com/patriciobos/Plantilla-CESE.git}. Los nombres de las carpetas son, o pretender ser, auto-explicativos.

\keyword{Appendices} -- Esta es la carpeta donde se deben poner los apéndices. Cada apéndice debe ir en su propio archivo \file{.tex}. Se incluye un ejemplo y una plantilla en la carpeta.

\keyword{Chapters} -- Esta es la carpeta donde se deben poner los capítulos de la memoria. Cada capítulo debe ir un su propio archivo \file{.tex} por separado.  Se ofrece por defecto, la siguiente estructura de capítulos y se recomienda su utilización dentro de lo posible:

\begin{itemize}
\item Capítulo 1: Introducción general	
\item Capítulo 2: Introducción específica
\item Capítulo 3: Diseño e implementación
\item Capítulo 4: Ensayos y resultados
\item Capítulo 5: Conclusiones

\end{itemize}

Esta estructura de capítulos es la que se recomienda para las memorias de la especialización.

\keyword{Figures} -- Esta carpeta contiene todas las figuras de la memoria.  Estas son las versiones finales de las imágenes que van a ser incluidas en la memoria.  Pueden ser imágenes en formato \textit{raster}\footnote{\url{https://en.wikipedia.org/wiki/Raster_graphics}} como \file{.png}, \file{.jpg} o en formato vectoriales\footnote{\url{https://en.wikipedia.org/wiki/Vector_graphics}} como \file{.pdf}, \file{.ps}.  Se debe notar que utilizar imágenes vectoriales disminuye notablemente el peso del documento final y acelera el tiempo de compilación por lo que es recomendable su utilización siempre que sea posible.

\subsection{Archivos}

También están incluidos varios archivos, la mayoría de ellos son de texto plano y se puede ver su contenido en un editor de texto. Después de la compilación inicial, se verá que más archivos auxiliares son creados por \ LaTeX{} o BibTeX, pero son de uso interno y no es necesario hacer nada en particular con ellos.  Toda la información necesaria para compilar el documento se encuentra en los archivos \file{.tex}, \file{.bib}, \file{.cls} y en las imágenes de la carpeta Figures.

\keyword{referencias.bib} - este es un archivo importante que contiene toda la información de referencias bibliográficas que se utilizarán para las citas en la memoria en conjunto con BibTeX. Usted puede escribir las entradas bibliográficas en forma manual, aunque existen también programas de gestión de referencias que facilitan la creación y gestión de las referencias y permiten exportarlas en formato BibTeX.  También hay disponibles sitios web como \url{books.google.com} que permiten obtener toda la información necesaria para una cita en formato BibTeX. Ver sección \ref{sec:biblio}

\keyword{MastersDoctoralThesis.cls} -- este es un archivo importante. Es el archivos con la clase que le informa a \LaTeX{} cómo debe dar formato a la memoria. El usuario de la plantilla no debería necesitar modificar nada de este archivo.

\keyword{memoria.pdf} -- esta es su memoria con una tipografía bellamente compuesta (en formato de archivo PDF) creado por \LaTeX{}. Se distribuye con la plantilla y después de compilar por primera vez sin hacer ningún cambio se debería obtener una versión idéntica a este documento.

\keyword{memoria.tex} -- este es un archivo importante. Este es el archivo que tiene que compilar \LaTeX{} para producir la memoria como un archivo PDF. Contiene un marco de trabajo y estructuras que le indican a \LaTeX{} cómo diagramar la memoria.  Está altamente comentado para que se pueda entender qué es lo que realiza cada línea de código y por qué está incluida en ese lugar.  En este archivo se debe completar la información personalizada de las primeras sección según se indica en la sección \ref{sec:FillingFile}.

Archivos que \emph{no} forman parte de la distribución de la plantilla pero que son generados por \LaTeX{} como archivos auxiliares necesarios para la producción de la memoria.pdf son:

\keyword{memoria.aux} -- este es un archivo auxiliar generado por \LaTeX{}, si se borra \LaTeX{} simplemente lo regenera cuando se compila el archivo principal \file{memoria.tex}.

\keyword{memoria.bbl} -- este es un archivo auxiliar generado por BibTeX, si se borra BibTeX simplemente lo regenera cuando se compila el archivo principal \file{memoria.tex}. Mientras que el archivo \file{.bib} contiene todas las referencias que hay, este archivo \file{.bbl} contine sólo las referencias que han sido citadas y se utiliza para la construcción de la bibiografía.

\keyword{memoria.blg} -- este es un archivo auxiliar generado por BibTeX, si se borra BibTeX simplemente lo regenera cuando se compila el archivo principal \file{memoria.tex}.

\keyword{memoria.lof} -- este es un archivo auxiliar generado por \LaTeX{}, si se borra \LaTeX{} simplemente lo regenera cuando se compila el archivo principal \file{memoria.tex}.  Le indica a \LaTeX{} cómo construir la sección \emph{Lista de Figuras}.
 
\keyword{memoria.log} --  este es un archivo auxiliar generado por \LaTeX{}, si se borra \LaTeX{} simplemente lo regenera cuando se compila el archivo principal \file{memoria.tex}. Contiene mensajes de \LaTeX{}. Si se reciben errores o advertencias durante la compilación, se guardan en este archivo \file{.log}.

\keyword{memoria.lot} -- este es un archivo auxiliar generado por \LaTeX{}, si se borra \LaTeX{} simplemente lo regenera cuando se compila el archivo principal \file{memoria.tex}.  Le indica a \LaTeX{} cómo construir la sección \emph{Lista de Tablas}.

\keyword{memoria.out} -- este es un archivo auxiliar generado por \LaTeX{}, si se borra \LaTeX{} simplemente lo regenera cuando se compila el archivo principal \file{memoria.tex}.

De esta larga lista de archivos, sólo aquellos con la extensión \file{.bib}, \file{.cls} y \file{.tex} son importantes.  Los otros archivos auxiliares pueden ser ignorados o borrados ya que \LaTeX{} y BibTeX los regenerarán durante la compilación.

%----------------------------------------------------------------------------------------

\section{Entorno de trabajo}

Ante de comenzar a editar la plantilla debemos tener un editor \LaTeX{} instalado en nuestra computadora.  En forma análoga a lo que sucede en lenguaje C, que se puede crear y editar código con casi cualquier editor, existen ciertos entornos de trabajo que nos pueden simplificar mucho la tarea.  En este sentido, se recomienda, sobre todo para los principiantes en \LaTeX{} la utilización de TexMaker, un programa gratuito y multi-plantaforma que está disponible tanto para windows como para sistemas GNU/linux.

La versión más reciente de TexMaker es la 4.5 y se puede descargar del siguiente link: \url{http://www.xm1math.net/texmaker/download.html}. Se puede consultar el manual de usuario en el siguiente link: \url{http://www.xm1math.net/texmaker/doc.html}.
 

\subsection{Paquetes adicionales}

Si bien durante el proceso de instalación de TexMaker, o cualquier otro editor que se haya elegido, se instalarán en el sistema los paquetes básicos necesarios para trabajar con \LaTeX{}, la plantilla de los trabajos de Especialización y Maestría requieren de paquete adicionales.

Se indican a continuación los comandos que se deben introducir en la consola de Ubuntu (ctrl + alt + t) para instalarlos:

\begin{lstlisting}[language=bash]
  $ sudo apt install texlive-lang-spanish texlive-science 
  $ sudo apt install texlive-bibtex-extra biber
  $ sudo apt install texlive texlive-fonts-recommended
  $ sudo apt install texlive-latex-extra
\end{lstlisting}


\subsection{Configurando TexMaker}


Una vez instalado el programa y los paquetes adicionales se debe abrir el archivo memoria.tex con el editor para ver una pantalla similar a la que se puede apreciar en la figura \ref{fig:texmaker}. 

\begin{figure}[h]
	\centering
	\includegraphics[width=\textwidth]{./Figures/texmaker.png}
	\caption{Entorno de trabajo de texMaker.}
	\label{fig:texmaker}
\end{figure}

Notar que existe una vista llamada Estructura a la izquierda de la interfaz que nos permite abrir desde dentro del programa los archivos individuales de los capítulos.  A la derecha se encuentra una vista con el archivo propiamente dicho para su edición. Hacia la parte inferior se encuentra una vista del log con información de los resultados de la compilación.  En esta última vista pueden aparecen advertencias o \textit{warning}, que normalmente pueden ser ignorados, y los errores que se indican en color rojo y deben resolverse para que se genere el PDF de salida.

Recordar que el archivo que se debe compilar con PDFLaTeX es \file{memoria.tex}, si se tratara de compilar alguno de los capítulos saldría un error.  Para salvar la molestia de tener que cambiar de archivo para compilar cada vez que se realice una modificación en un capítulo, se puede definir el archivo \file{memoria.tex} como ``documento maestro'' yendo al menú opciones -> ``definir documento actual como documento maestro'', lo que permite compilar con PDFLaTeX memoria.tex directamente desde cualquier archivo que se esté modificando . Se muestra esta opción en la figura \ref{fig:docMaestro}.

\begin{figure}[h]
	\centering
	\includegraphics[width=\textwidth]{./Figures/docMaestro.png}
	\caption{Definir memoria.tex como documento maestro.}
	\label{fig:docMaestro}
\end{figure}

En el menú herramientas se encuentran las opciones de compilación.  Para producir un archivo PDF a partir de un archivo .tex se debe ejecutar PDFLaTeX (el shortcut es F6). Para incorporar nueva bibliografía se debe utilizar la opción BibTeX del mismo menú herramientas (el shortcut es F11).

Notar que para actualizar las tablas de contenidos se debe ejecutar PDFLaTeX dos veces.  Esto se debe a que es necesario actualizar algunos archivos auxiliares antes de obtener el resultado final.  En forma similar, para actualizar las referencias se debe ejecutar primero PDFLaTeX, después BibTeX y finalmente PDFLaTeX dos veces por idénticos motivos.

\section{Personalizando la plantilla, el archivo \file{memoria.tex}}
\label{sec:FillingFile}

Para personalizar la plantilla se debe incorporar la información propia en los distintos archivos \file{.tex}. 

Primero abrir \file{memoria.tex} con TexMaker (o el editor de su preferencia). Se debe ubicar dentro del archivo el bloque de código titulado \emph{INFORMACIÓN DE LA PORTADA} donde se deben incorporar los primeros datos personales con los que se construirá automáticamente la portada.


%----------------------------------------------------------------------------------------

\section{El código del archivo \file{memoria.tex} explicado}

El archivo \file{memoria.tex} contiene la estructura del documento y es el archivo de mayor jerarquía de la memoria.  Podría ser equiparable a la función \emph{main()} de un programa en C, o mejor dicho al archivo fuente .c donde se encuentra definida la función main().

La estructura básica de cualquier documento de \LaTeX{} comienza con la definición de clase del documento, es seguida por un preámbulo donde se pueden agregar funcionalidades con el uso de \texttt{paquetes} (equiparables a bibliotecas de C), y finalmente, termina con el cuerpo del documento, donde irá el contenido de la memoria.

\lstset{%
  basicstyle=\small\ttfamily,
  language=[LaTeX]{TeX}
}

\begin{lstlisting}
\documentclass{article}  <- Definicion de clase
\usepackage{listings}	 <- Preambulo

\begin{document}	 <- Comienzo del contenido propio 
	Hello world!
\end{document}
\end{lstlisting}


El archivo \file{memoria.tex} se encuentra densamente comentado para explicar qué páginas, secciones y elementos de formato está creando el código \LaTeX{} en cada línea. El código está dividido en bloques con nombres en mayúsculas para que resulte evidente qué es lo que hace esa porción de código en particular. Inicialmente puede parecer que hay mucho código \LaTeX{}, pero es principalmente código para dar formato a la memoria por lo que no requiere intervención del usuario de la plantilla.  Sí se deben personalizar con su información los bloques indicados como:

\begin{itemize}
	\item Informacion de la memoria
	\item Resumen
	\item Agradecimientos
	\item Dedicatoria
\end{itemize}

El índice de contenidos, las listas de figura de tablas se generan en forma automática y no requieren intervención ni edición manual por parte del usuario de la plantilla. 

En la parte final del documento se encuentran los capítulos y los apéndices.  Por defecto se incluyen los 5 capítulos propuestos que se encuentran en la carpeta /Chapters. Cada capítulo se debe escribir en un archivo .tex separado y se debe poner en la carpeta \emph{Chapters} con el nombre \file{Chapter1}, \file{Chapter2}, etc\ldots El código para incluir capítulos desde archivos externos se muestra a continuación.

\begin{verbatim}
	% Chapter 1

\chapter{Introducción general} % Main chapter title

\label{Chapter1} % For referencing the chapter elsewhere, use \ref{Chapter1} 
\label{IntroGeneral}
En este capítulo se presentan las características de los robots de servicio, se  reseña el uso de luz ultravioleta como germicida y se exponen los objetivos que motivaron el presente trabajo y sus respectivo alcance.
%----------------------------------------------------------------------------------------

% Define some commands to keep the formatting separated from the content 
\newcommand{\keyword}[1]{\textbf{#1}}
\newcommand{\tabhead}[1]{\textbf{#1}}
\newcommand{\code}[1]{\texttt{#1}}
\newcommand{\file}[1]{\texttt{\bfseries#1}}
\newcommand{\option}[1]{\texttt{\itshape#1}}
\newcommand{\grados}{$^{\circ}$}

%----------------------------------------------------------------------------------------

%\section{Introducción}

%----------------------------------------------------------------------------------------
\section{Robots de servicio}

A lo largo del siglo XX la robótica pasó de ser una temática de la rama de la ciencia ficción, a cumplir un importante rol dentro de los complejos industriales. En los últimos años los robots han pasado a tener cada vez más tareas de “servicio” para ambientes  públicos y hogareños.
La robótica de servicios abarca un amplio campo de aplicaciones, la mayoría de las cuales tienen diferentes grados de automatización, desde la teleoperación completa hasta la funcionamiento autónomo, y constituye un campo de aplicación más diverso que el de la robótica industrial. En la  figura \ref{fig:robotsservicio} se pueden observar tres tipos de robots de servicios: una aspiradora hogareña, un cortador de césped y un limpiavidrios.

\begin{figure}[h]
	\centering
	\includegraphics[width=\textwidth]{./Figures/robotsservicio.jpg}
	\caption{robots de servicio.\protect\footnotemark.}
	\label{fig:robotsservicio}
\end{figure}
\footnotetext{Imágenes tomadas de \url{https://www.domotizar.com/}}


A mediados de la década de 1990, la Comisión Económica de las Naciones Unidas para Europa (UNECE) .\citep{UNECE} y la Federación Internacional de Robótica (IFR) .\citep{IFR} adoptaron un sistema de clasificación de robots de servicio dividida por categorías y tipos de interacción, que se ha mantenido clasificación actual. En la  figura \ref{fig:robotsservicio} se puede observar los primeros ítems de clasificación para robots domésticos/personales de acuerdo a los tipos y áreas de aplicación.



\begin{figure}[h]
	\centering
	\includegraphics[width=\textwidth]{./Figures/clasificacion.png}
	\caption{clasificación de robots de servicio.\protect\footnotemark.}
	\label{fig:clasificacion}
\end{figure}
\footnotetext{Imagen tomada de \url{https://www.editores-srl.com.ar/sites/default/files/aa1_ifr_robots.pdf}}




\subsection{Robots móviles para inspección y limpieza}

Los robots móviles son dispositivos que poseen un sistema de locomoción capaz de navegar a través de un determinado ambiente de trabajo. Normalmente cuentan con cierto nivel de autonomía que les permite el desplazamiento sin colisiones por un recorrido específico. Sus aplicaciones son muchas y en general  están relacionadas con tareas monótonas o riesgosas para la salud humana.
Las plataformas móviles pueden realizar tareas de inspección y limpieza de manera autónoma o controlada remotamente por un operador. Son utilizadas en zonas de difícil acceso debido a limitaciones de espacio o razones de seguridad.  
Este tipo de robot suele contar con sensores de distinto tipo, para detectar los límites y obstáculos ante los que se presentan. 
La proliferación de de robots para limpieza se incrementó fuertemente a partir de la pandemia de Covid-19, con lo que se los puede encontrar hoy en día en espacios en los que antes no estaban presentes, tales como salas médicas,  hoteles y en el transporte público  .\citep{Cleaning}. Estos dispositivos “de interior” abarcan a la aspiradoras robóticas y a los robots de lavado de pisos que limpian pisos con funciones de barrido y trapeado húmedo. 


\subsubsection{Robots de limpieza UVC}

Acá va una comparativa de robots de limpieza UVC.
.
.
.
.
.


%----------------------------------------------------------------------------------------

\section{Desinfección usando Luz ultravioleta}

El espectro ultravioleta (UV) abarca la banda de radiación electromagnética entre los 400 y 100 nm, presentando una longitud de onda menor que la de la luz visible y mayor que la de los rayos X.  Se divide en tres las siguientes categorías principales: los rayos UV-A (400 – 315 nm), que son los más cercanos al espectro visible; los rayos UV-B (315 – 280 nm), que son absorbidos en gran parte por diferentes elementos a medida que atraviesan el cielo y los rayos UV-C (280 – 200 nm), que son absorbidos totalmente por la capa de ozono. En la  figura \ref{fig:espectro} se observa detalle de parte del espectro de radiación electromagnética y  su clasificación según longitud de onda.


\begin{figure}[h]
	\centering
	\includegraphics[width=\textwidth]{./Figures/espectro.PNG}
	\caption{clasificación según longitud de onda.\protect\footnotemark.}
	\label{fig:espectro}
\end{figure}
\footnotetext{Imagen tomada de \url{https://www.lit-uv.com/es/technology/}}

La utilización de luz ultravioleta UV-C como germicida ha demostrado efectividad para la esterilización  las bacterias, gérmenes, virus, algas y esporas. 

Los virus tienen un tamaño inferior a un micrómetro (µm, una millonésima parte de un metro) y las bacterias son típicamente de 0,5 a 5 µm. Técnicamente es incorrecto decir que los rayos  UV-C matan a los virus, siendo que no se trata de organismos vivientes. Sin embargo, el comité de foto-biología de la Illuminating Engineering Society (IES) informa que los fotones UV-C interactúan con el ARN y las moléculas de ADN en un virus o bacteria de modo que se evita su reproducción y por lo tanto su efecto infeccioso. A este proceso se lo denomina “desactivación”  .\citep{IES}.

La International Ultraviolet Association (IUVA) afirma que los resultados de pruebas en laboratorio de desinfección utilizando UV-C entre los 200 y 280nm demuestran especial utilidad para reducir la transmisión de los virus causantes del COVID-19:  SARS-CoV-1 y MERS-CoV .\citep{IUA}. En la práctica, el efecto depende de factores tales como  el tiempo de exposición y obstrucción que puedan tener los rayos para alcanzar plenamente los pliegues u ondulaciones que pudiera tener la superficie a desinfectar. 

Este tipo de desinfección, que no genera residuos químicos, es especialmente recomendada cuando debe realizarse sobre materiales que podrían verse afectados o dañados ante la limpieza continua con productos a base de alcoholes líquidos, como ser dispositivos electrónicos o materiales susceptibles de oxidación. También es especialmente aplicable en el caso de superficies de difícil acceso por su ubicación o por presentar formas y estructuras que no permiten la higienización por contacto con paños o rociadores. 
Por otra parte, si bien la Organización Mundial de la Salud (OMS) recomienda el uso de rayos UV-C para desinfección, también alerta sobre los riesgos de  exposición en seres humanos y animales, cuya piel puede verse irritada, a la vez que puede producir daños a la vista [6]. En este sentido promueven la limpieza de manos periódica con jabón o con alcohol, y dejan la esterilización con UV-C para  instrumental y objetos de uso diario.


%----------------------------------------------------------------------------------------

\section{Qué incluye esta plantilla}

\subsection{Carpetas}

Esta plantilla se distribuye como una único archivo .zip que se puede descomprimir en varios archivos y carpetas. Asimismo, se puede consultar el repositorio git para obtener la última versión de los archivos, \url{https://github.com/patriciobos/Plantilla-CESE.git}. Los nombres de las carpetas son, o pretender ser, auto-explicativos.

\keyword{Appendices} -- Esta es la carpeta donde se deben poner los apéndices. Cada apéndice debe ir en su propio archivo \file{.tex}. Se incluye un ejemplo y una plantilla en la carpeta.

\keyword{Chapters} -- Esta es la carpeta donde se deben poner los capítulos de la memoria. Cada capítulo debe ir un su propio archivo \file{.tex} por separado.  Se ofrece por defecto, la siguiente estructura de capítulos y se recomienda su utilización dentro de lo posible:

\begin{itemize}
\item Capítulo 1: Introducción general	
\item Capítulo 2: Introducción específica
\item Capítulo 3: Diseño e implementación
\item Capítulo 4: Ensayos y resultados
\item Capítulo 5: Conclusiones

\end{itemize}

Esta estructura de capítulos es la que se recomienda para las memorias de la especialización.

\keyword{Figures} -- Esta carpeta contiene todas las figuras de la memoria.  Estas son las versiones finales de las imágenes que van a ser incluidas en la memoria.  Pueden ser imágenes en formato \textit{raster}\footnote{\url{https://en.wikipedia.org/wiki/Raster_graphics}} como \file{.png}, \file{.jpg} o en formato vectoriales\footnote{\url{https://en.wikipedia.org/wiki/Vector_graphics}} como \file{.pdf}, \file{.ps}.  Se debe notar que utilizar imágenes vectoriales disminuye notablemente el peso del documento final y acelera el tiempo de compilación por lo que es recomendable su utilización siempre que sea posible.

\subsection{Archivos}

También están incluidos varios archivos, la mayoría de ellos son de texto plano y se puede ver su contenido en un editor de texto. Después de la compilación inicial, se verá que más archivos auxiliares son creados por \ LaTeX{} o BibTeX, pero son de uso interno y no es necesario hacer nada en particular con ellos.  Toda la información necesaria para compilar el documento se encuentra en los archivos \file{.tex}, \file{.bib}, \file{.cls} y en las imágenes de la carpeta Figures.

\keyword{referencias.bib} - este es un archivo importante que contiene toda la información de referencias bibliográficas que se utilizarán para las citas en la memoria en conjunto con BibTeX. Usted puede escribir las entradas bibliográficas en forma manual, aunque existen también programas de gestión de referencias que facilitan la creación y gestión de las referencias y permiten exportarlas en formato BibTeX.  También hay disponibles sitios web como \url{books.google.com} que permiten obtener toda la información necesaria para una cita en formato BibTeX. Ver sección \ref{sec:biblio}

\keyword{MastersDoctoralThesis.cls} -- este es un archivo importante. Es el archivos con la clase que le informa a \LaTeX{} cómo debe dar formato a la memoria. El usuario de la plantilla no debería necesitar modificar nada de este archivo.

\keyword{memoria.pdf} -- esta es su memoria con una tipografía bellamente compuesta (en formato de archivo PDF) creado por \LaTeX{}. Se distribuye con la plantilla y después de compilar por primera vez sin hacer ningún cambio se debería obtener una versión idéntica a este documento.

\keyword{memoria.tex} -- este es un archivo importante. Este es el archivo que tiene que compilar \LaTeX{} para producir la memoria como un archivo PDF. Contiene un marco de trabajo y estructuras que le indican a \LaTeX{} cómo diagramar la memoria.  Está altamente comentado para que se pueda entender qué es lo que realiza cada línea de código y por qué está incluida en ese lugar.  En este archivo se debe completar la información personalizada de las primeras sección según se indica en la sección \ref{sec:FillingFile}.

Archivos que \emph{no} forman parte de la distribución de la plantilla pero que son generados por \LaTeX{} como archivos auxiliares necesarios para la producción de la memoria.pdf son:

\keyword{memoria.aux} -- este es un archivo auxiliar generado por \LaTeX{}, si se borra \LaTeX{} simplemente lo regenera cuando se compila el archivo principal \file{memoria.tex}.

\keyword{memoria.bbl} -- este es un archivo auxiliar generado por BibTeX, si se borra BibTeX simplemente lo regenera cuando se compila el archivo principal \file{memoria.tex}. Mientras que el archivo \file{.bib} contiene todas las referencias que hay, este archivo \file{.bbl} contine sólo las referencias que han sido citadas y se utiliza para la construcción de la bibiografía.

\keyword{memoria.blg} -- este es un archivo auxiliar generado por BibTeX, si se borra BibTeX simplemente lo regenera cuando se compila el archivo principal \file{memoria.tex}.

\keyword{memoria.lof} -- este es un archivo auxiliar generado por \LaTeX{}, si se borra \LaTeX{} simplemente lo regenera cuando se compila el archivo principal \file{memoria.tex}.  Le indica a \LaTeX{} cómo construir la sección \emph{Lista de Figuras}.
 
\keyword{memoria.log} --  este es un archivo auxiliar generado por \LaTeX{}, si se borra \LaTeX{} simplemente lo regenera cuando se compila el archivo principal \file{memoria.tex}. Contiene mensajes de \LaTeX{}. Si se reciben errores o advertencias durante la compilación, se guardan en este archivo \file{.log}.

\keyword{memoria.lot} -- este es un archivo auxiliar generado por \LaTeX{}, si se borra \LaTeX{} simplemente lo regenera cuando se compila el archivo principal \file{memoria.tex}.  Le indica a \LaTeX{} cómo construir la sección \emph{Lista de Tablas}.

\keyword{memoria.out} -- este es un archivo auxiliar generado por \LaTeX{}, si se borra \LaTeX{} simplemente lo regenera cuando se compila el archivo principal \file{memoria.tex}.

De esta larga lista de archivos, sólo aquellos con la extensión \file{.bib}, \file{.cls} y \file{.tex} son importantes.  Los otros archivos auxiliares pueden ser ignorados o borrados ya que \LaTeX{} y BibTeX los regenerarán durante la compilación.

%----------------------------------------------------------------------------------------

\section{Entorno de trabajo}

Ante de comenzar a editar la plantilla debemos tener un editor \LaTeX{} instalado en nuestra computadora.  En forma análoga a lo que sucede en lenguaje C, que se puede crear y editar código con casi cualquier editor, existen ciertos entornos de trabajo que nos pueden simplificar mucho la tarea.  En este sentido, se recomienda, sobre todo para los principiantes en \LaTeX{} la utilización de TexMaker, un programa gratuito y multi-plantaforma que está disponible tanto para windows como para sistemas GNU/linux.

La versión más reciente de TexMaker es la 4.5 y se puede descargar del siguiente link: \url{http://www.xm1math.net/texmaker/download.html}. Se puede consultar el manual de usuario en el siguiente link: \url{http://www.xm1math.net/texmaker/doc.html}.
 

\subsection{Paquetes adicionales}

Si bien durante el proceso de instalación de TexMaker, o cualquier otro editor que se haya elegido, se instalarán en el sistema los paquetes básicos necesarios para trabajar con \LaTeX{}, la plantilla de los trabajos de Especialización y Maestría requieren de paquete adicionales.

Se indican a continuación los comandos que se deben introducir en la consola de Ubuntu (ctrl + alt + t) para instalarlos:

\begin{lstlisting}[language=bash]
  $ sudo apt install texlive-lang-spanish texlive-science 
  $ sudo apt install texlive-bibtex-extra biber
  $ sudo apt install texlive texlive-fonts-recommended
  $ sudo apt install texlive-latex-extra
\end{lstlisting}


\subsection{Configurando TexMaker}


Una vez instalado el programa y los paquetes adicionales se debe abrir el archivo memoria.tex con el editor para ver una pantalla similar a la que se puede apreciar en la figura \ref{fig:texmaker}. 

\begin{figure}[h]
	\centering
	\includegraphics[width=\textwidth]{./Figures/texmaker.png}
	\caption{Entorno de trabajo de texMaker.}
	\label{fig:texmaker}
\end{figure}

Notar que existe una vista llamada Estructura a la izquierda de la interfaz que nos permite abrir desde dentro del programa los archivos individuales de los capítulos.  A la derecha se encuentra una vista con el archivo propiamente dicho para su edición. Hacia la parte inferior se encuentra una vista del log con información de los resultados de la compilación.  En esta última vista pueden aparecen advertencias o \textit{warning}, que normalmente pueden ser ignorados, y los errores que se indican en color rojo y deben resolverse para que se genere el PDF de salida.

Recordar que el archivo que se debe compilar con PDFLaTeX es \file{memoria.tex}, si se tratara de compilar alguno de los capítulos saldría un error.  Para salvar la molestia de tener que cambiar de archivo para compilar cada vez que se realice una modificación en un capítulo, se puede definir el archivo \file{memoria.tex} como ``documento maestro'' yendo al menú opciones -> ``definir documento actual como documento maestro'', lo que permite compilar con PDFLaTeX memoria.tex directamente desde cualquier archivo que se esté modificando . Se muestra esta opción en la figura \ref{fig:docMaestro}.

\begin{figure}[h]
	\centering
	\includegraphics[width=\textwidth]{./Figures/docMaestro.png}
	\caption{Definir memoria.tex como documento maestro.}
	\label{fig:docMaestro}
\end{figure}

En el menú herramientas se encuentran las opciones de compilación.  Para producir un archivo PDF a partir de un archivo .tex se debe ejecutar PDFLaTeX (el shortcut es F6). Para incorporar nueva bibliografía se debe utilizar la opción BibTeX del mismo menú herramientas (el shortcut es F11).

Notar que para actualizar las tablas de contenidos se debe ejecutar PDFLaTeX dos veces.  Esto se debe a que es necesario actualizar algunos archivos auxiliares antes de obtener el resultado final.  En forma similar, para actualizar las referencias se debe ejecutar primero PDFLaTeX, después BibTeX y finalmente PDFLaTeX dos veces por idénticos motivos.

\section{Personalizando la plantilla, el archivo \file{memoria.tex}}
\label{sec:FillingFile}

Para personalizar la plantilla se debe incorporar la información propia en los distintos archivos \file{.tex}. 

Primero abrir \file{memoria.tex} con TexMaker (o el editor de su preferencia). Se debe ubicar dentro del archivo el bloque de código titulado \emph{INFORMACIÓN DE LA PORTADA} donde se deben incorporar los primeros datos personales con los que se construirá automáticamente la portada.


%----------------------------------------------------------------------------------------

\section{El código del archivo \file{memoria.tex} explicado}

El archivo \file{memoria.tex} contiene la estructura del documento y es el archivo de mayor jerarquía de la memoria.  Podría ser equiparable a la función \emph{main()} de un programa en C, o mejor dicho al archivo fuente .c donde se encuentra definida la función main().

La estructura básica de cualquier documento de \LaTeX{} comienza con la definición de clase del documento, es seguida por un preámbulo donde se pueden agregar funcionalidades con el uso de \texttt{paquetes} (equiparables a bibliotecas de C), y finalmente, termina con el cuerpo del documento, donde irá el contenido de la memoria.

\lstset{%
  basicstyle=\small\ttfamily,
  language=[LaTeX]{TeX}
}

\begin{lstlisting}
\documentclass{article}  <- Definicion de clase
\usepackage{listings}	 <- Preambulo

\begin{document}	 <- Comienzo del contenido propio 
	Hello world!
\end{document}
\end{lstlisting}


El archivo \file{memoria.tex} se encuentra densamente comentado para explicar qué páginas, secciones y elementos de formato está creando el código \LaTeX{} en cada línea. El código está dividido en bloques con nombres en mayúsculas para que resulte evidente qué es lo que hace esa porción de código en particular. Inicialmente puede parecer que hay mucho código \LaTeX{}, pero es principalmente código para dar formato a la memoria por lo que no requiere intervención del usuario de la plantilla.  Sí se deben personalizar con su información los bloques indicados como:

\begin{itemize}
	\item Informacion de la memoria
	\item Resumen
	\item Agradecimientos
	\item Dedicatoria
\end{itemize}

El índice de contenidos, las listas de figura de tablas se generan en forma automática y no requieren intervención ni edición manual por parte del usuario de la plantilla. 

En la parte final del documento se encuentran los capítulos y los apéndices.  Por defecto se incluyen los 5 capítulos propuestos que se encuentran en la carpeta /Chapters. Cada capítulo se debe escribir en un archivo .tex separado y se debe poner en la carpeta \emph{Chapters} con el nombre \file{Chapter1}, \file{Chapter2}, etc\ldots El código para incluir capítulos desde archivos externos se muestra a continuación.

\begin{verbatim}
	% Chapter 1

\chapter{Introducción general} % Main chapter title

\label{Chapter1} % For referencing the chapter elsewhere, use \ref{Chapter1} 
\label{IntroGeneral}
En este capítulo se presentan las características de los robots de servicio, se  reseña el uso de luz ultravioleta como germicida y se exponen los objetivos que motivaron el presente trabajo y sus respectivo alcance.
%----------------------------------------------------------------------------------------

% Define some commands to keep the formatting separated from the content 
\newcommand{\keyword}[1]{\textbf{#1}}
\newcommand{\tabhead}[1]{\textbf{#1}}
\newcommand{\code}[1]{\texttt{#1}}
\newcommand{\file}[1]{\texttt{\bfseries#1}}
\newcommand{\option}[1]{\texttt{\itshape#1}}
\newcommand{\grados}{$^{\circ}$}

%----------------------------------------------------------------------------------------

%\section{Introducción}

%----------------------------------------------------------------------------------------
\section{Robots de servicio}

A lo largo del siglo XX la robótica pasó de ser una temática de la rama de la ciencia ficción, a cumplir un importante rol dentro de los complejos industriales. En los últimos años los robots han pasado a tener cada vez más tareas de “servicio” para ambientes  públicos y hogareños.
La robótica de servicios abarca un amplio campo de aplicaciones, la mayoría de las cuales tienen diferentes grados de automatización, desde la teleoperación completa hasta la funcionamiento autónomo, y constituye un campo de aplicación más diverso que el de la robótica industrial. En la  figura \ref{fig:robotsservicio} se pueden observar tres tipos de robots de servicios: una aspiradora hogareña, un cortador de césped y un limpiavidrios.

\begin{figure}[h]
	\centering
	\includegraphics[width=\textwidth]{./Figures/robotsservicio.jpg}
	\caption{robots de servicio.\protect\footnotemark.}
	\label{fig:robotsservicio}
\end{figure}
\footnotetext{Imágenes tomadas de \url{https://www.domotizar.com/}}


A mediados de la década de 1990, la Comisión Económica de las Naciones Unidas para Europa (UNECE) .\citep{UNECE} y la Federación Internacional de Robótica (IFR) .\citep{IFR} adoptaron un sistema de clasificación de robots de servicio dividida por categorías y tipos de interacción, que se ha mantenido clasificación actual. En la  figura \ref{fig:robotsservicio} se puede observar los primeros ítems de clasificación para robots domésticos/personales de acuerdo a los tipos y áreas de aplicación.



\begin{figure}[h]
	\centering
	\includegraphics[width=\textwidth]{./Figures/clasificacion.png}
	\caption{clasificación de robots de servicio.\protect\footnotemark.}
	\label{fig:clasificacion}
\end{figure}
\footnotetext{Imagen tomada de \url{https://www.editores-srl.com.ar/sites/default/files/aa1_ifr_robots.pdf}}




\subsection{Robots móviles para inspección y limpieza}

Los robots móviles son dispositivos que poseen un sistema de locomoción capaz de navegar a través de un determinado ambiente de trabajo. Normalmente cuentan con cierto nivel de autonomía que les permite el desplazamiento sin colisiones por un recorrido específico. Sus aplicaciones son muchas y en general  están relacionadas con tareas monótonas o riesgosas para la salud humana.
Las plataformas móviles pueden realizar tareas de inspección y limpieza de manera autónoma o controlada remotamente por un operador. Son utilizadas en zonas de difícil acceso debido a limitaciones de espacio o razones de seguridad.  
Este tipo de robot suele contar con sensores de distinto tipo, para detectar los límites y obstáculos ante los que se presentan. 
La proliferación de de robots para limpieza se incrementó fuertemente a partir de la pandemia de Covid-19, con lo que se los puede encontrar hoy en día en espacios en los que antes no estaban presentes, tales como salas médicas,  hoteles y en el transporte público  .\citep{Cleaning}. Estos dispositivos “de interior” abarcan a la aspiradoras robóticas y a los robots de lavado de pisos que limpian pisos con funciones de barrido y trapeado húmedo. 


\subsubsection{Robots de limpieza UVC}

Acá va una comparativa de robots de limpieza UVC.
.
.
.
.
.


%----------------------------------------------------------------------------------------

\section{Desinfección usando Luz ultravioleta}

El espectro ultravioleta (UV) abarca la banda de radiación electromagnética entre los 400 y 100 nm, presentando una longitud de onda menor que la de la luz visible y mayor que la de los rayos X.  Se divide en tres las siguientes categorías principales: los rayos UV-A (400 – 315 nm), que son los más cercanos al espectro visible; los rayos UV-B (315 – 280 nm), que son absorbidos en gran parte por diferentes elementos a medida que atraviesan el cielo y los rayos UV-C (280 – 200 nm), que son absorbidos totalmente por la capa de ozono. En la  figura \ref{fig:espectro} se observa detalle de parte del espectro de radiación electromagnética y  su clasificación según longitud de onda.


\begin{figure}[h]
	\centering
	\includegraphics[width=\textwidth]{./Figures/espectro.PNG}
	\caption{clasificación según longitud de onda.\protect\footnotemark.}
	\label{fig:espectro}
\end{figure}
\footnotetext{Imagen tomada de \url{https://www.lit-uv.com/es/technology/}}

La utilización de luz ultravioleta UV-C como germicida ha demostrado efectividad para la esterilización  las bacterias, gérmenes, virus, algas y esporas. 

Los virus tienen un tamaño inferior a un micrómetro (µm, una millonésima parte de un metro) y las bacterias son típicamente de 0,5 a 5 µm. Técnicamente es incorrecto decir que los rayos  UV-C matan a los virus, siendo que no se trata de organismos vivientes. Sin embargo, el comité de foto-biología de la Illuminating Engineering Society (IES) informa que los fotones UV-C interactúan con el ARN y las moléculas de ADN en un virus o bacteria de modo que se evita su reproducción y por lo tanto su efecto infeccioso. A este proceso se lo denomina “desactivación”  .\citep{IES}.

La International Ultraviolet Association (IUVA) afirma que los resultados de pruebas en laboratorio de desinfección utilizando UV-C entre los 200 y 280nm demuestran especial utilidad para reducir la transmisión de los virus causantes del COVID-19:  SARS-CoV-1 y MERS-CoV .\citep{IUA}. En la práctica, el efecto depende de factores tales como  el tiempo de exposición y obstrucción que puedan tener los rayos para alcanzar plenamente los pliegues u ondulaciones que pudiera tener la superficie a desinfectar. 

Este tipo de desinfección, que no genera residuos químicos, es especialmente recomendada cuando debe realizarse sobre materiales que podrían verse afectados o dañados ante la limpieza continua con productos a base de alcoholes líquidos, como ser dispositivos electrónicos o materiales susceptibles de oxidación. También es especialmente aplicable en el caso de superficies de difícil acceso por su ubicación o por presentar formas y estructuras que no permiten la higienización por contacto con paños o rociadores. 
Por otra parte, si bien la Organización Mundial de la Salud (OMS) recomienda el uso de rayos UV-C para desinfección, también alerta sobre los riesgos de  exposición en seres humanos y animales, cuya piel puede verse irritada, a la vez que puede producir daños a la vista [6]. En este sentido promueven la limpieza de manos periódica con jabón o con alcohol, y dejan la esterilización con UV-C para  instrumental y objetos de uso diario.


%----------------------------------------------------------------------------------------

\section{Qué incluye esta plantilla}

\subsection{Carpetas}

Esta plantilla se distribuye como una único archivo .zip que se puede descomprimir en varios archivos y carpetas. Asimismo, se puede consultar el repositorio git para obtener la última versión de los archivos, \url{https://github.com/patriciobos/Plantilla-CESE.git}. Los nombres de las carpetas son, o pretender ser, auto-explicativos.

\keyword{Appendices} -- Esta es la carpeta donde se deben poner los apéndices. Cada apéndice debe ir en su propio archivo \file{.tex}. Se incluye un ejemplo y una plantilla en la carpeta.

\keyword{Chapters} -- Esta es la carpeta donde se deben poner los capítulos de la memoria. Cada capítulo debe ir un su propio archivo \file{.tex} por separado.  Se ofrece por defecto, la siguiente estructura de capítulos y se recomienda su utilización dentro de lo posible:

\begin{itemize}
\item Capítulo 1: Introducción general	
\item Capítulo 2: Introducción específica
\item Capítulo 3: Diseño e implementación
\item Capítulo 4: Ensayos y resultados
\item Capítulo 5: Conclusiones

\end{itemize}

Esta estructura de capítulos es la que se recomienda para las memorias de la especialización.

\keyword{Figures} -- Esta carpeta contiene todas las figuras de la memoria.  Estas son las versiones finales de las imágenes que van a ser incluidas en la memoria.  Pueden ser imágenes en formato \textit{raster}\footnote{\url{https://en.wikipedia.org/wiki/Raster_graphics}} como \file{.png}, \file{.jpg} o en formato vectoriales\footnote{\url{https://en.wikipedia.org/wiki/Vector_graphics}} como \file{.pdf}, \file{.ps}.  Se debe notar que utilizar imágenes vectoriales disminuye notablemente el peso del documento final y acelera el tiempo de compilación por lo que es recomendable su utilización siempre que sea posible.

\subsection{Archivos}

También están incluidos varios archivos, la mayoría de ellos son de texto plano y se puede ver su contenido en un editor de texto. Después de la compilación inicial, se verá que más archivos auxiliares son creados por \ LaTeX{} o BibTeX, pero son de uso interno y no es necesario hacer nada en particular con ellos.  Toda la información necesaria para compilar el documento se encuentra en los archivos \file{.tex}, \file{.bib}, \file{.cls} y en las imágenes de la carpeta Figures.

\keyword{referencias.bib} - este es un archivo importante que contiene toda la información de referencias bibliográficas que se utilizarán para las citas en la memoria en conjunto con BibTeX. Usted puede escribir las entradas bibliográficas en forma manual, aunque existen también programas de gestión de referencias que facilitan la creación y gestión de las referencias y permiten exportarlas en formato BibTeX.  También hay disponibles sitios web como \url{books.google.com} que permiten obtener toda la información necesaria para una cita en formato BibTeX. Ver sección \ref{sec:biblio}

\keyword{MastersDoctoralThesis.cls} -- este es un archivo importante. Es el archivos con la clase que le informa a \LaTeX{} cómo debe dar formato a la memoria. El usuario de la plantilla no debería necesitar modificar nada de este archivo.

\keyword{memoria.pdf} -- esta es su memoria con una tipografía bellamente compuesta (en formato de archivo PDF) creado por \LaTeX{}. Se distribuye con la plantilla y después de compilar por primera vez sin hacer ningún cambio se debería obtener una versión idéntica a este documento.

\keyword{memoria.tex} -- este es un archivo importante. Este es el archivo que tiene que compilar \LaTeX{} para producir la memoria como un archivo PDF. Contiene un marco de trabajo y estructuras que le indican a \LaTeX{} cómo diagramar la memoria.  Está altamente comentado para que se pueda entender qué es lo que realiza cada línea de código y por qué está incluida en ese lugar.  En este archivo se debe completar la información personalizada de las primeras sección según se indica en la sección \ref{sec:FillingFile}.

Archivos que \emph{no} forman parte de la distribución de la plantilla pero que son generados por \LaTeX{} como archivos auxiliares necesarios para la producción de la memoria.pdf son:

\keyword{memoria.aux} -- este es un archivo auxiliar generado por \LaTeX{}, si se borra \LaTeX{} simplemente lo regenera cuando se compila el archivo principal \file{memoria.tex}.

\keyword{memoria.bbl} -- este es un archivo auxiliar generado por BibTeX, si se borra BibTeX simplemente lo regenera cuando se compila el archivo principal \file{memoria.tex}. Mientras que el archivo \file{.bib} contiene todas las referencias que hay, este archivo \file{.bbl} contine sólo las referencias que han sido citadas y se utiliza para la construcción de la bibiografía.

\keyword{memoria.blg} -- este es un archivo auxiliar generado por BibTeX, si se borra BibTeX simplemente lo regenera cuando se compila el archivo principal \file{memoria.tex}.

\keyword{memoria.lof} -- este es un archivo auxiliar generado por \LaTeX{}, si se borra \LaTeX{} simplemente lo regenera cuando se compila el archivo principal \file{memoria.tex}.  Le indica a \LaTeX{} cómo construir la sección \emph{Lista de Figuras}.
 
\keyword{memoria.log} --  este es un archivo auxiliar generado por \LaTeX{}, si se borra \LaTeX{} simplemente lo regenera cuando se compila el archivo principal \file{memoria.tex}. Contiene mensajes de \LaTeX{}. Si se reciben errores o advertencias durante la compilación, se guardan en este archivo \file{.log}.

\keyword{memoria.lot} -- este es un archivo auxiliar generado por \LaTeX{}, si se borra \LaTeX{} simplemente lo regenera cuando se compila el archivo principal \file{memoria.tex}.  Le indica a \LaTeX{} cómo construir la sección \emph{Lista de Tablas}.

\keyword{memoria.out} -- este es un archivo auxiliar generado por \LaTeX{}, si se borra \LaTeX{} simplemente lo regenera cuando se compila el archivo principal \file{memoria.tex}.

De esta larga lista de archivos, sólo aquellos con la extensión \file{.bib}, \file{.cls} y \file{.tex} son importantes.  Los otros archivos auxiliares pueden ser ignorados o borrados ya que \LaTeX{} y BibTeX los regenerarán durante la compilación.

%----------------------------------------------------------------------------------------

\section{Entorno de trabajo}

Ante de comenzar a editar la plantilla debemos tener un editor \LaTeX{} instalado en nuestra computadora.  En forma análoga a lo que sucede en lenguaje C, que se puede crear y editar código con casi cualquier editor, existen ciertos entornos de trabajo que nos pueden simplificar mucho la tarea.  En este sentido, se recomienda, sobre todo para los principiantes en \LaTeX{} la utilización de TexMaker, un programa gratuito y multi-plantaforma que está disponible tanto para windows como para sistemas GNU/linux.

La versión más reciente de TexMaker es la 4.5 y se puede descargar del siguiente link: \url{http://www.xm1math.net/texmaker/download.html}. Se puede consultar el manual de usuario en el siguiente link: \url{http://www.xm1math.net/texmaker/doc.html}.
 

\subsection{Paquetes adicionales}

Si bien durante el proceso de instalación de TexMaker, o cualquier otro editor que se haya elegido, se instalarán en el sistema los paquetes básicos necesarios para trabajar con \LaTeX{}, la plantilla de los trabajos de Especialización y Maestría requieren de paquete adicionales.

Se indican a continuación los comandos que se deben introducir en la consola de Ubuntu (ctrl + alt + t) para instalarlos:

\begin{lstlisting}[language=bash]
  $ sudo apt install texlive-lang-spanish texlive-science 
  $ sudo apt install texlive-bibtex-extra biber
  $ sudo apt install texlive texlive-fonts-recommended
  $ sudo apt install texlive-latex-extra
\end{lstlisting}


\subsection{Configurando TexMaker}


Una vez instalado el programa y los paquetes adicionales se debe abrir el archivo memoria.tex con el editor para ver una pantalla similar a la que se puede apreciar en la figura \ref{fig:texmaker}. 

\begin{figure}[h]
	\centering
	\includegraphics[width=\textwidth]{./Figures/texmaker.png}
	\caption{Entorno de trabajo de texMaker.}
	\label{fig:texmaker}
\end{figure}

Notar que existe una vista llamada Estructura a la izquierda de la interfaz que nos permite abrir desde dentro del programa los archivos individuales de los capítulos.  A la derecha se encuentra una vista con el archivo propiamente dicho para su edición. Hacia la parte inferior se encuentra una vista del log con información de los resultados de la compilación.  En esta última vista pueden aparecen advertencias o \textit{warning}, que normalmente pueden ser ignorados, y los errores que se indican en color rojo y deben resolverse para que se genere el PDF de salida.

Recordar que el archivo que se debe compilar con PDFLaTeX es \file{memoria.tex}, si se tratara de compilar alguno de los capítulos saldría un error.  Para salvar la molestia de tener que cambiar de archivo para compilar cada vez que se realice una modificación en un capítulo, se puede definir el archivo \file{memoria.tex} como ``documento maestro'' yendo al menú opciones -> ``definir documento actual como documento maestro'', lo que permite compilar con PDFLaTeX memoria.tex directamente desde cualquier archivo que se esté modificando . Se muestra esta opción en la figura \ref{fig:docMaestro}.

\begin{figure}[h]
	\centering
	\includegraphics[width=\textwidth]{./Figures/docMaestro.png}
	\caption{Definir memoria.tex como documento maestro.}
	\label{fig:docMaestro}
\end{figure}

En el menú herramientas se encuentran las opciones de compilación.  Para producir un archivo PDF a partir de un archivo .tex se debe ejecutar PDFLaTeX (el shortcut es F6). Para incorporar nueva bibliografía se debe utilizar la opción BibTeX del mismo menú herramientas (el shortcut es F11).

Notar que para actualizar las tablas de contenidos se debe ejecutar PDFLaTeX dos veces.  Esto se debe a que es necesario actualizar algunos archivos auxiliares antes de obtener el resultado final.  En forma similar, para actualizar las referencias se debe ejecutar primero PDFLaTeX, después BibTeX y finalmente PDFLaTeX dos veces por idénticos motivos.

\section{Personalizando la plantilla, el archivo \file{memoria.tex}}
\label{sec:FillingFile}

Para personalizar la plantilla se debe incorporar la información propia en los distintos archivos \file{.tex}. 

Primero abrir \file{memoria.tex} con TexMaker (o el editor de su preferencia). Se debe ubicar dentro del archivo el bloque de código titulado \emph{INFORMACIÓN DE LA PORTADA} donde se deben incorporar los primeros datos personales con los que se construirá automáticamente la portada.


%----------------------------------------------------------------------------------------

\section{El código del archivo \file{memoria.tex} explicado}

El archivo \file{memoria.tex} contiene la estructura del documento y es el archivo de mayor jerarquía de la memoria.  Podría ser equiparable a la función \emph{main()} de un programa en C, o mejor dicho al archivo fuente .c donde se encuentra definida la función main().

La estructura básica de cualquier documento de \LaTeX{} comienza con la definición de clase del documento, es seguida por un preámbulo donde se pueden agregar funcionalidades con el uso de \texttt{paquetes} (equiparables a bibliotecas de C), y finalmente, termina con el cuerpo del documento, donde irá el contenido de la memoria.

\lstset{%
  basicstyle=\small\ttfamily,
  language=[LaTeX]{TeX}
}

\begin{lstlisting}
\documentclass{article}  <- Definicion de clase
\usepackage{listings}	 <- Preambulo

\begin{document}	 <- Comienzo del contenido propio 
	Hello world!
\end{document}
\end{lstlisting}


El archivo \file{memoria.tex} se encuentra densamente comentado para explicar qué páginas, secciones y elementos de formato está creando el código \LaTeX{} en cada línea. El código está dividido en bloques con nombres en mayúsculas para que resulte evidente qué es lo que hace esa porción de código en particular. Inicialmente puede parecer que hay mucho código \LaTeX{}, pero es principalmente código para dar formato a la memoria por lo que no requiere intervención del usuario de la plantilla.  Sí se deben personalizar con su información los bloques indicados como:

\begin{itemize}
	\item Informacion de la memoria
	\item Resumen
	\item Agradecimientos
	\item Dedicatoria
\end{itemize}

El índice de contenidos, las listas de figura de tablas se generan en forma automática y no requieren intervención ni edición manual por parte del usuario de la plantilla. 

En la parte final del documento se encuentran los capítulos y los apéndices.  Por defecto se incluyen los 5 capítulos propuestos que se encuentran en la carpeta /Chapters. Cada capítulo se debe escribir en un archivo .tex separado y se debe poner en la carpeta \emph{Chapters} con el nombre \file{Chapter1}, \file{Chapter2}, etc\ldots El código para incluir capítulos desde archivos externos se muestra a continuación.

\begin{verbatim}
	% Chapter 1

\chapter{Introducción general} % Main chapter title

\label{Chapter1} % For referencing the chapter elsewhere, use \ref{Chapter1} 
\label{IntroGeneral}
En este capítulo se presentan las características de los robots de servicio, se  reseña el uso de luz ultravioleta como germicida y se exponen los objetivos que motivaron el presente trabajo y sus respectivo alcance.
%----------------------------------------------------------------------------------------

% Define some commands to keep the formatting separated from the content 
\newcommand{\keyword}[1]{\textbf{#1}}
\newcommand{\tabhead}[1]{\textbf{#1}}
\newcommand{\code}[1]{\texttt{#1}}
\newcommand{\file}[1]{\texttt{\bfseries#1}}
\newcommand{\option}[1]{\texttt{\itshape#1}}
\newcommand{\grados}{$^{\circ}$}

%----------------------------------------------------------------------------------------

%\section{Introducción}

%----------------------------------------------------------------------------------------
\section{Robots de servicio}

A lo largo del siglo XX la robótica pasó de ser una temática de la rama de la ciencia ficción, a cumplir un importante rol dentro de los complejos industriales. En los últimos años los robots han pasado a tener cada vez más tareas de “servicio” para ambientes  públicos y hogareños.
La robótica de servicios abarca un amplio campo de aplicaciones, la mayoría de las cuales tienen diferentes grados de automatización, desde la teleoperación completa hasta la funcionamiento autónomo, y constituye un campo de aplicación más diverso que el de la robótica industrial. En la  figura \ref{fig:robotsservicio} se pueden observar tres tipos de robots de servicios: una aspiradora hogareña, un cortador de césped y un limpiavidrios.

\begin{figure}[h]
	\centering
	\includegraphics[width=\textwidth]{./Figures/robotsservicio.jpg}
	\caption{robots de servicio.\protect\footnotemark.}
	\label{fig:robotsservicio}
\end{figure}
\footnotetext{Imágenes tomadas de \url{https://www.domotizar.com/}}


A mediados de la década de 1990, la Comisión Económica de las Naciones Unidas para Europa (UNECE) .\citep{UNECE} y la Federación Internacional de Robótica (IFR) .\citep{IFR} adoptaron un sistema de clasificación de robots de servicio dividida por categorías y tipos de interacción, que se ha mantenido clasificación actual. En la  figura \ref{fig:robotsservicio} se puede observar los primeros ítems de clasificación para robots domésticos/personales de acuerdo a los tipos y áreas de aplicación.



\begin{figure}[h]
	\centering
	\includegraphics[width=\textwidth]{./Figures/clasificacion.png}
	\caption{clasificación de robots de servicio.\protect\footnotemark.}
	\label{fig:clasificacion}
\end{figure}
\footnotetext{Imagen tomada de \url{https://www.editores-srl.com.ar/sites/default/files/aa1_ifr_robots.pdf}}




\subsection{Robots móviles para inspección y limpieza}

Los robots móviles son dispositivos que poseen un sistema de locomoción capaz de navegar a través de un determinado ambiente de trabajo. Normalmente cuentan con cierto nivel de autonomía que les permite el desplazamiento sin colisiones por un recorrido específico. Sus aplicaciones son muchas y en general  están relacionadas con tareas monótonas o riesgosas para la salud humana.
Las plataformas móviles pueden realizar tareas de inspección y limpieza de manera autónoma o controlada remotamente por un operador. Son utilizadas en zonas de difícil acceso debido a limitaciones de espacio o razones de seguridad.  
Este tipo de robot suele contar con sensores de distinto tipo, para detectar los límites y obstáculos ante los que se presentan. 
La proliferación de de robots para limpieza se incrementó fuertemente a partir de la pandemia de Covid-19, con lo que se los puede encontrar hoy en día en espacios en los que antes no estaban presentes, tales como salas médicas,  hoteles y en el transporte público  .\citep{Cleaning}. Estos dispositivos “de interior” abarcan a la aspiradoras robóticas y a los robots de lavado de pisos que limpian pisos con funciones de barrido y trapeado húmedo. 


\subsubsection{Robots de limpieza UVC}

Acá va una comparativa de robots de limpieza UVC.
.
.
.
.
.


%----------------------------------------------------------------------------------------

\section{Desinfección usando Luz ultravioleta}

El espectro ultravioleta (UV) abarca la banda de radiación electromagnética entre los 400 y 100 nm, presentando una longitud de onda menor que la de la luz visible y mayor que la de los rayos X.  Se divide en tres las siguientes categorías principales: los rayos UV-A (400 – 315 nm), que son los más cercanos al espectro visible; los rayos UV-B (315 – 280 nm), que son absorbidos en gran parte por diferentes elementos a medida que atraviesan el cielo y los rayos UV-C (280 – 200 nm), que son absorbidos totalmente por la capa de ozono. En la  figura \ref{fig:espectro} se observa detalle de parte del espectro de radiación electromagnética y  su clasificación según longitud de onda.


\begin{figure}[h]
	\centering
	\includegraphics[width=\textwidth]{./Figures/espectro.PNG}
	\caption{clasificación según longitud de onda.\protect\footnotemark.}
	\label{fig:espectro}
\end{figure}
\footnotetext{Imagen tomada de \url{https://www.lit-uv.com/es/technology/}}

La utilización de luz ultravioleta UV-C como germicida ha demostrado efectividad para la esterilización  las bacterias, gérmenes, virus, algas y esporas. 

Los virus tienen un tamaño inferior a un micrómetro (µm, una millonésima parte de un metro) y las bacterias son típicamente de 0,5 a 5 µm. Técnicamente es incorrecto decir que los rayos  UV-C matan a los virus, siendo que no se trata de organismos vivientes. Sin embargo, el comité de foto-biología de la Illuminating Engineering Society (IES) informa que los fotones UV-C interactúan con el ARN y las moléculas de ADN en un virus o bacteria de modo que se evita su reproducción y por lo tanto su efecto infeccioso. A este proceso se lo denomina “desactivación”  .\citep{IES}.

La International Ultraviolet Association (IUVA) afirma que los resultados de pruebas en laboratorio de desinfección utilizando UV-C entre los 200 y 280nm demuestran especial utilidad para reducir la transmisión de los virus causantes del COVID-19:  SARS-CoV-1 y MERS-CoV .\citep{IUA}. En la práctica, el efecto depende de factores tales como  el tiempo de exposición y obstrucción que puedan tener los rayos para alcanzar plenamente los pliegues u ondulaciones que pudiera tener la superficie a desinfectar. 

Este tipo de desinfección, que no genera residuos químicos, es especialmente recomendada cuando debe realizarse sobre materiales que podrían verse afectados o dañados ante la limpieza continua con productos a base de alcoholes líquidos, como ser dispositivos electrónicos o materiales susceptibles de oxidación. También es especialmente aplicable en el caso de superficies de difícil acceso por su ubicación o por presentar formas y estructuras que no permiten la higienización por contacto con paños o rociadores. 
Por otra parte, si bien la Organización Mundial de la Salud (OMS) recomienda el uso de rayos UV-C para desinfección, también alerta sobre los riesgos de  exposición en seres humanos y animales, cuya piel puede verse irritada, a la vez que puede producir daños a la vista [6]. En este sentido promueven la limpieza de manos periódica con jabón o con alcohol, y dejan la esterilización con UV-C para  instrumental y objetos de uso diario.


%----------------------------------------------------------------------------------------

\section{Qué incluye esta plantilla}

\subsection{Carpetas}

Esta plantilla se distribuye como una único archivo .zip que se puede descomprimir en varios archivos y carpetas. Asimismo, se puede consultar el repositorio git para obtener la última versión de los archivos, \url{https://github.com/patriciobos/Plantilla-CESE.git}. Los nombres de las carpetas son, o pretender ser, auto-explicativos.

\keyword{Appendices} -- Esta es la carpeta donde se deben poner los apéndices. Cada apéndice debe ir en su propio archivo \file{.tex}. Se incluye un ejemplo y una plantilla en la carpeta.

\keyword{Chapters} -- Esta es la carpeta donde se deben poner los capítulos de la memoria. Cada capítulo debe ir un su propio archivo \file{.tex} por separado.  Se ofrece por defecto, la siguiente estructura de capítulos y se recomienda su utilización dentro de lo posible:

\begin{itemize}
\item Capítulo 1: Introducción general	
\item Capítulo 2: Introducción específica
\item Capítulo 3: Diseño e implementación
\item Capítulo 4: Ensayos y resultados
\item Capítulo 5: Conclusiones

\end{itemize}

Esta estructura de capítulos es la que se recomienda para las memorias de la especialización.

\keyword{Figures} -- Esta carpeta contiene todas las figuras de la memoria.  Estas son las versiones finales de las imágenes que van a ser incluidas en la memoria.  Pueden ser imágenes en formato \textit{raster}\footnote{\url{https://en.wikipedia.org/wiki/Raster_graphics}} como \file{.png}, \file{.jpg} o en formato vectoriales\footnote{\url{https://en.wikipedia.org/wiki/Vector_graphics}} como \file{.pdf}, \file{.ps}.  Se debe notar que utilizar imágenes vectoriales disminuye notablemente el peso del documento final y acelera el tiempo de compilación por lo que es recomendable su utilización siempre que sea posible.

\subsection{Archivos}

También están incluidos varios archivos, la mayoría de ellos son de texto plano y se puede ver su contenido en un editor de texto. Después de la compilación inicial, se verá que más archivos auxiliares son creados por \ LaTeX{} o BibTeX, pero son de uso interno y no es necesario hacer nada en particular con ellos.  Toda la información necesaria para compilar el documento se encuentra en los archivos \file{.tex}, \file{.bib}, \file{.cls} y en las imágenes de la carpeta Figures.

\keyword{referencias.bib} - este es un archivo importante que contiene toda la información de referencias bibliográficas que se utilizarán para las citas en la memoria en conjunto con BibTeX. Usted puede escribir las entradas bibliográficas en forma manual, aunque existen también programas de gestión de referencias que facilitan la creación y gestión de las referencias y permiten exportarlas en formato BibTeX.  También hay disponibles sitios web como \url{books.google.com} que permiten obtener toda la información necesaria para una cita en formato BibTeX. Ver sección \ref{sec:biblio}

\keyword{MastersDoctoralThesis.cls} -- este es un archivo importante. Es el archivos con la clase que le informa a \LaTeX{} cómo debe dar formato a la memoria. El usuario de la plantilla no debería necesitar modificar nada de este archivo.

\keyword{memoria.pdf} -- esta es su memoria con una tipografía bellamente compuesta (en formato de archivo PDF) creado por \LaTeX{}. Se distribuye con la plantilla y después de compilar por primera vez sin hacer ningún cambio se debería obtener una versión idéntica a este documento.

\keyword{memoria.tex} -- este es un archivo importante. Este es el archivo que tiene que compilar \LaTeX{} para producir la memoria como un archivo PDF. Contiene un marco de trabajo y estructuras que le indican a \LaTeX{} cómo diagramar la memoria.  Está altamente comentado para que se pueda entender qué es lo que realiza cada línea de código y por qué está incluida en ese lugar.  En este archivo se debe completar la información personalizada de las primeras sección según se indica en la sección \ref{sec:FillingFile}.

Archivos que \emph{no} forman parte de la distribución de la plantilla pero que son generados por \LaTeX{} como archivos auxiliares necesarios para la producción de la memoria.pdf son:

\keyword{memoria.aux} -- este es un archivo auxiliar generado por \LaTeX{}, si se borra \LaTeX{} simplemente lo regenera cuando se compila el archivo principal \file{memoria.tex}.

\keyword{memoria.bbl} -- este es un archivo auxiliar generado por BibTeX, si se borra BibTeX simplemente lo regenera cuando se compila el archivo principal \file{memoria.tex}. Mientras que el archivo \file{.bib} contiene todas las referencias que hay, este archivo \file{.bbl} contine sólo las referencias que han sido citadas y se utiliza para la construcción de la bibiografía.

\keyword{memoria.blg} -- este es un archivo auxiliar generado por BibTeX, si se borra BibTeX simplemente lo regenera cuando se compila el archivo principal \file{memoria.tex}.

\keyword{memoria.lof} -- este es un archivo auxiliar generado por \LaTeX{}, si se borra \LaTeX{} simplemente lo regenera cuando se compila el archivo principal \file{memoria.tex}.  Le indica a \LaTeX{} cómo construir la sección \emph{Lista de Figuras}.
 
\keyword{memoria.log} --  este es un archivo auxiliar generado por \LaTeX{}, si se borra \LaTeX{} simplemente lo regenera cuando se compila el archivo principal \file{memoria.tex}. Contiene mensajes de \LaTeX{}. Si se reciben errores o advertencias durante la compilación, se guardan en este archivo \file{.log}.

\keyword{memoria.lot} -- este es un archivo auxiliar generado por \LaTeX{}, si se borra \LaTeX{} simplemente lo regenera cuando se compila el archivo principal \file{memoria.tex}.  Le indica a \LaTeX{} cómo construir la sección \emph{Lista de Tablas}.

\keyword{memoria.out} -- este es un archivo auxiliar generado por \LaTeX{}, si se borra \LaTeX{} simplemente lo regenera cuando se compila el archivo principal \file{memoria.tex}.

De esta larga lista de archivos, sólo aquellos con la extensión \file{.bib}, \file{.cls} y \file{.tex} son importantes.  Los otros archivos auxiliares pueden ser ignorados o borrados ya que \LaTeX{} y BibTeX los regenerarán durante la compilación.

%----------------------------------------------------------------------------------------

\section{Entorno de trabajo}

Ante de comenzar a editar la plantilla debemos tener un editor \LaTeX{} instalado en nuestra computadora.  En forma análoga a lo que sucede en lenguaje C, que se puede crear y editar código con casi cualquier editor, existen ciertos entornos de trabajo que nos pueden simplificar mucho la tarea.  En este sentido, se recomienda, sobre todo para los principiantes en \LaTeX{} la utilización de TexMaker, un programa gratuito y multi-plantaforma que está disponible tanto para windows como para sistemas GNU/linux.

La versión más reciente de TexMaker es la 4.5 y se puede descargar del siguiente link: \url{http://www.xm1math.net/texmaker/download.html}. Se puede consultar el manual de usuario en el siguiente link: \url{http://www.xm1math.net/texmaker/doc.html}.
 

\subsection{Paquetes adicionales}

Si bien durante el proceso de instalación de TexMaker, o cualquier otro editor que se haya elegido, se instalarán en el sistema los paquetes básicos necesarios para trabajar con \LaTeX{}, la plantilla de los trabajos de Especialización y Maestría requieren de paquete adicionales.

Se indican a continuación los comandos que se deben introducir en la consola de Ubuntu (ctrl + alt + t) para instalarlos:

\begin{lstlisting}[language=bash]
  $ sudo apt install texlive-lang-spanish texlive-science 
  $ sudo apt install texlive-bibtex-extra biber
  $ sudo apt install texlive texlive-fonts-recommended
  $ sudo apt install texlive-latex-extra
\end{lstlisting}


\subsection{Configurando TexMaker}


Una vez instalado el programa y los paquetes adicionales se debe abrir el archivo memoria.tex con el editor para ver una pantalla similar a la que se puede apreciar en la figura \ref{fig:texmaker}. 

\begin{figure}[h]
	\centering
	\includegraphics[width=\textwidth]{./Figures/texmaker.png}
	\caption{Entorno de trabajo de texMaker.}
	\label{fig:texmaker}
\end{figure}

Notar que existe una vista llamada Estructura a la izquierda de la interfaz que nos permite abrir desde dentro del programa los archivos individuales de los capítulos.  A la derecha se encuentra una vista con el archivo propiamente dicho para su edición. Hacia la parte inferior se encuentra una vista del log con información de los resultados de la compilación.  En esta última vista pueden aparecen advertencias o \textit{warning}, que normalmente pueden ser ignorados, y los errores que se indican en color rojo y deben resolverse para que se genere el PDF de salida.

Recordar que el archivo que se debe compilar con PDFLaTeX es \file{memoria.tex}, si se tratara de compilar alguno de los capítulos saldría un error.  Para salvar la molestia de tener que cambiar de archivo para compilar cada vez que se realice una modificación en un capítulo, se puede definir el archivo \file{memoria.tex} como ``documento maestro'' yendo al menú opciones -> ``definir documento actual como documento maestro'', lo que permite compilar con PDFLaTeX memoria.tex directamente desde cualquier archivo que se esté modificando . Se muestra esta opción en la figura \ref{fig:docMaestro}.

\begin{figure}[h]
	\centering
	\includegraphics[width=\textwidth]{./Figures/docMaestro.png}
	\caption{Definir memoria.tex como documento maestro.}
	\label{fig:docMaestro}
\end{figure}

En el menú herramientas se encuentran las opciones de compilación.  Para producir un archivo PDF a partir de un archivo .tex se debe ejecutar PDFLaTeX (el shortcut es F6). Para incorporar nueva bibliografía se debe utilizar la opción BibTeX del mismo menú herramientas (el shortcut es F11).

Notar que para actualizar las tablas de contenidos se debe ejecutar PDFLaTeX dos veces.  Esto se debe a que es necesario actualizar algunos archivos auxiliares antes de obtener el resultado final.  En forma similar, para actualizar las referencias se debe ejecutar primero PDFLaTeX, después BibTeX y finalmente PDFLaTeX dos veces por idénticos motivos.

\section{Personalizando la plantilla, el archivo \file{memoria.tex}}
\label{sec:FillingFile}

Para personalizar la plantilla se debe incorporar la información propia en los distintos archivos \file{.tex}. 

Primero abrir \file{memoria.tex} con TexMaker (o el editor de su preferencia). Se debe ubicar dentro del archivo el bloque de código titulado \emph{INFORMACIÓN DE LA PORTADA} donde se deben incorporar los primeros datos personales con los que se construirá automáticamente la portada.


%----------------------------------------------------------------------------------------

\section{El código del archivo \file{memoria.tex} explicado}

El archivo \file{memoria.tex} contiene la estructura del documento y es el archivo de mayor jerarquía de la memoria.  Podría ser equiparable a la función \emph{main()} de un programa en C, o mejor dicho al archivo fuente .c donde se encuentra definida la función main().

La estructura básica de cualquier documento de \LaTeX{} comienza con la definición de clase del documento, es seguida por un preámbulo donde se pueden agregar funcionalidades con el uso de \texttt{paquetes} (equiparables a bibliotecas de C), y finalmente, termina con el cuerpo del documento, donde irá el contenido de la memoria.

\lstset{%
  basicstyle=\small\ttfamily,
  language=[LaTeX]{TeX}
}

\begin{lstlisting}
\documentclass{article}  <- Definicion de clase
\usepackage{listings}	 <- Preambulo

\begin{document}	 <- Comienzo del contenido propio 
	Hello world!
\end{document}
\end{lstlisting}


El archivo \file{memoria.tex} se encuentra densamente comentado para explicar qué páginas, secciones y elementos de formato está creando el código \LaTeX{} en cada línea. El código está dividido en bloques con nombres en mayúsculas para que resulte evidente qué es lo que hace esa porción de código en particular. Inicialmente puede parecer que hay mucho código \LaTeX{}, pero es principalmente código para dar formato a la memoria por lo que no requiere intervención del usuario de la plantilla.  Sí se deben personalizar con su información los bloques indicados como:

\begin{itemize}
	\item Informacion de la memoria
	\item Resumen
	\item Agradecimientos
	\item Dedicatoria
\end{itemize}

El índice de contenidos, las listas de figura de tablas se generan en forma automática y no requieren intervención ni edición manual por parte del usuario de la plantilla. 

En la parte final del documento se encuentran los capítulos y los apéndices.  Por defecto se incluyen los 5 capítulos propuestos que se encuentran en la carpeta /Chapters. Cada capítulo se debe escribir en un archivo .tex separado y se debe poner en la carpeta \emph{Chapters} con el nombre \file{Chapter1}, \file{Chapter2}, etc\ldots El código para incluir capítulos desde archivos externos se muestra a continuación.

\begin{verbatim}
	\include{Chapters/Chapter1}
	\include{Chapters/Chapter2} 
	\include{Chapters/Chapter3}
	\include{Chapters/Chapter4} 
	\include{Chapters/Chapter5} 
\end{verbatim}

Los apéndices también deben escribirse en archivos .tex separados, que se deben ubicar dentro de la carpeta \emph{Appendices}. Los apéndices vienen comentados por defecto con el caracter \code{\%} y para incluirlos simplemente se debe eliminar dicho caracter.

Finalmente, se encuentra el código para incluir la bibliografía en el documento final.  Este código tampoco debe modificarse. La metodología para trabajar las referencias bibliográficas se desarrolla en la sección \ref{sec:biblio}.
%----------------------------------------------------------------------------------------

\section{Bibliografía}
\label{sec:biblio}

Las opciones de formato de la bibliografía se controlan a través del paquete de latex \option{biblatex} que se incluye en la memoria en el archivo memoria.tex.  Estas opciones determinan cómo se generan las citas bibliográficas en el cuerpo del documento y cómo se genera la bibliografía al final de la memoria.

En el preámbulo se puede encontrar el código que incluye el paquete biblatex, que no requiere ninguna modificación del usuario de la plantilla, y que contiene las siguientes opciones:

\begin{lstlisting}
\usepackage[backend=bibtex,
	natbib=true, 
	style=numeric, 
	sorting=none]
{biblatex}
\end{lstlisting}

En el archivo \file{reference.bib} se encuentran las referencias bibliográficas que se pueden citar en el documento.  Para incorporar una nueva cita al documento lo primero es agregarla en este archivo con todos los campos necesario.  Todas las entradas bibliográficas comienzan con $@$ y una palabra que define el formato de la entrada.  Para cada formato existen campos obligatorios que deben completarse. No importa el orden en que las entradas estén definidas en el archivo .bib.  Tampoco es importante el orden en que estén definidos los campos de una entrada bibliográfica. A continuación se muestran algunos ejemplos:

\begin{lstlisting}
@ARTICLE{ARTICLE:1,
    AUTHOR="John Doe",
    TITLE="Title",
    JOURNAL="Journal",
    YEAR="2017",
}
\end{lstlisting}


\begin{lstlisting}
@BOOK{BOOK:1,
    AUTHOR="John Doe",
    TITLE="The Book without Title",
    PUBLISHER="Dummy Publisher",
    YEAR="2100",
}
\end{lstlisting}


\begin{lstlisting}
@INBOOK{BOOK:2,
    AUTHOR="John Doe",
    TITLE="The Book without Title",
    PUBLISHER="Dummy Publisher",
    YEAR="2100",
    PAGES="100-200",
}
\end{lstlisting}


\begin{lstlisting}
@MISC{WEBSITE:1,
    HOWPUBLISHED = "\url{http://example.com}",
    AUTHOR = "Intel",
    TITLE = "Example Website",
    MONTH = "12",
    YEAR = "1988",
    URLDATE = {2012-11-26}
}
\end{lstlisting}

Se debe notar que los nombres \emph{ARTICLE:1}, \emph{BOOK:1}, \emph{BOOK:2} y \emph{WEBSITE:1} son nombres de fantasía que le sirve al autor del documento para identificar la entrada. En este sentido, se podrían reemplazar por cualquier otro nombre.  Tampoco es necesario poner : seguido de un número, en los ejemplos sólo se incluye como un posible estilo para identificar las entradas.

La entradas se citan en el documento con el comando: 

\begin{verbatim}
\citep{nombre_de_la_entrada}
\end{verbatim}

Y cuando se usan, se muestran así: \citep{ARTICLE:1}, \citep{BOOK:1}, \citep{BOOK:2}, \citep{WEBSITE:1}.  Notar cómo se conforma la sección Bibliografía al final del documento. 

	\chapter{Introducción específica} % Main chapter title

\label{Chapter2}

%----------------------------------------------------------------------------------------
%	SECTION 1
%----------------------------------------------------------------------------------------
En este capítulo se presentan las distintas tecnologías y metodologías disponibles para la implementación del prototipo de robot móvil. Se describen los dispositivos y arquitecturas más significativos que permitieron alcanzar los requerimientos planteados.

\section{Criterios de diseño del robot}

En esta sección se enumeran los aspectos considerados a la hora diseñar el robot. Se tuvieron en cuenta los alcances establecidos como así también las posibilidades económicas de solventar el proyecto. 



\subsection{Placa de microprocesamiento}
Se utilizó  la placa de desarrollo EDUCIAA-NXP \citep{EDUCIAA} ya que la misma se usa para la ejercitación en varias asignaturas de la carrera de postgrado. En la figura \ref{fig:EDUCIAANXP} se observa una imagen de la EDU-CIAA-NXP, que es una versión de bajo costo de la CIAA-NXP, pensada para la enseñanza universitaria, terciaria y secundaria. 

\begin{figure}[htpb]
	\centering
	\includegraphics[width=\textwidth]{./Figures/EDUCIAANXP.jpg}
	\caption{placa de desarrollo EDUCIAA-NXP.\protect\footnotemark.}
	\label{fig:EDUCIAANXP}
\end{figure}
\footnotetext{Imagen tomada de \url{http://www.proyecto-ciaa.com.ar}}



\begin{figure}[htpb]
	\centering
	\includegraphics[width=9cm]{./Figures/Bloques.jpg}
	\caption{Diagrama en bloques de la EDUCIAA-NXP.\protect\footnotemark.}
	\label{fig:Bloques}
\end{figure}
\footnotetext{Imagen tomada de \url{http://www.proyecto-ciaa.com.ar}}


En la figura \ref{fig:Bloques} puede verse un diagrama en bloques general de la placa. 

El  microcontrolador utilizado por la EDU-CIAA es el LPC4337 (dual core ARM Cortex-M4F y Cortex-M0). Los recursos más significativos que se utilizaron de la placa fueron: 


\begin{itemize}
	\item GPIO (General Purpose Input/Output, Entrada/Salida de Propósito General)
	\item PWM (Pulse Width Modulation, modulación por ancho de pulso).
	\item UART (Universal Asynchronous Receiver-Transmitter, Transmisor-Receptor Asíncrono Universal).
	\item Temporizadores.
\end{itemize}



\subsection{Driver de motores}

Se utilizó un módulo para el accionamiento de motores \citep{Driver}. En la figura \ref{fig:Driver} se puede observar una imagen de la placa 

\begin{figure}[h]
	\centering
	\includegraphics[width=5cm]{./Figures/L298N.png}
	\caption{Driver de motores.\protect\footnotemark.}
	\label{fig:Driver}
\end{figure}
\footnotetext{Imagen tomada de \url{http://robots-argentina.com.ar}}


El módulo está basado en el circuito integrado L298N \citep{L298} y permite controlar dos motores de corriente continua de manera simultánea e independiente.  Sus características principales son:

\begin{itemize}
	\item Tensión mínima: 5 V.
	\item Tensión máxima: 35 V.
	\item Corriente máxima: 2 A.
	\item Tensión de nivel lógico: 5 V.
	\item Potencia máxima 25 W
	\item Medidas: 43 x 43 x 24 mm
\end{itemize}

La placa tiene la opción de habilitar o no el regulador LM7805 integrado para alimentar la parte lógica. En la figura \ref{fig:Esquema} se observa el diagrama esquemático del módulo.


\begin{figure}[h]
	\centering
	\includegraphics[width=12cm]{./Figures/Modulo.png}
	\caption{diagrama esquemático del módulo.\protect\footnotemark.}
	\label{fig:Esquema}
\end{figure}
\footnotetext{Imagen tomada de \url{http://robots-argentina.com.ar}}
\pagebreak



\subsection{Módulo sensor de Infrarrojos}

Se utilizaron dos módulos sensores de proximidad por infrarrojos IR FC-51 \citep{IR} para la detección de obstáculos por parte del robot. Estos módulos están compuestos por  un emisor de luz infrarroja (IR)  y un receptor que detecta su reflejo en  las superficies contra las que se enfrenta, de modo que presentan una señal en  presencia de cualquier obstáculo en su parte frontal. Un potenciómetro permite al ajustar el rango de detección. 

El sensor presenta una respuesta estable incluso con luz ambiente o en completa oscuridad. En la figura \ref{fig:moduloIR} se observa una imagen del sensor de Infrarrojos.

\begin{figure}[h]
	\centering
	\includegraphics[width=8cm]{./Figures/moduloIR.jpg}
	\caption{Módulo sensor de Infrarrojos .\protect\footnotemark.}
	\label{fig:moduloIR}
\end{figure}
\footnotetext{Imagen tomada de \url{http://robots-argentina.com.ar}}



En la figura \ref{fig:IRschem} se muestra el circuito esquemático  del sensor de Infrarrojos

\begin{figure}[h]
	\centering
	\includegraphics[width=14cm]{./Figures/IRschem.jpg}
	\caption{Esquema del módulo sensor de Infrarrojos .\protect\footnotemark.}
	\label{fig:IRschem}
\end{figure}
\footnotetext{Imagen tomada de \url{http://robots-argentina.com.ar}}
\pagebreak

Las características del módulo son:
\begin{itemize}
	\item Ángulo de cobertura: 35°.
	\item Tensión de funcionamiento: 3 V – 6 V.
	\item Rango de detección: 2 cm – 30 cm (ajustable con el potenciómetro).
	\item Tamaño: 4,5 cm x 1,4 cm x 0.7cm. 
	\item Discriminación: Las salida toma nivel lógico bajo cuando se detecta un obstáculo (reflexión).
\end{itemize}



\subsection{Baterías}

En función del consumo y la intensión de no dedicar mayor espacio a las celdas de alimentación, se emplearon dos baterías de Ion-litio tipo 18650. Una de las ventajas de las Ion-litio es que permiten ser recargadas con una media de entre 600 a 1000 veces sin que se estropeen ni pierdan efectividad \citep{18650}. La capacidad de estas baterías varían de un modelo a otro pero suelen estar comprendidas entre los 2100 maH y los 4000maH.Su tensión nominal es de 3,7 V (hasta 4,2 V en vacío).

Las baterías van conectadas en serie para lograr una tensión de 7,4 V acorde a la alimentación de los motores y con un margen superior necesario para el correcto funcionamiento del regulador de tensión de 5 V del módulo de accionamiento de motores. 
Las baterías se insertaron en un portapilas comercial. En la figura \ref{fig:portapila} se muestra las dos baterías 18650 ya instaladas en su portapila.


\begin{figure}[h]
	\centering
	\includegraphics[width=10cm]{./Figures/portapilas.PNG}
	\caption{Las dos baterías 18650 en su portapila .}
	\label{fig:portapila}
\end{figure}


\subsection{Módulo de comunicaciones Bluetooth}

Se utilizó el módulo Bluetooth HC-05 para la comunicación comunicaciones con el robot \citep{HC05}. El mismo ya había sido utilizado en  prácticas de la asignatura “Protocolos de Comunicaciones en sistemas embebidos”, conectado al puerto serie de la EDU-CIAA. Todos los parámetros del módulo se pueden configurar mediante comandos AT. 

En la figura \ref{fig:moduloHC05} se observa el módulo HC-05.


\begin{figure}[h]
	\centering
	\includegraphics[width=8cm]{./Figures/HC05.jpeg}
	\caption{Módulo Bluetooth HC-05 .\protect\footnotemark.}
	\label{fig:moduloHC05}
\end{figure}
\footnotetext{Imagen tomada de \url{https://maker.pro/custom/tutorial/hc-05-bluetooth-transceiver-module-datasheet-highlights}}



Las características del módulo son:
\begin{itemize}
	\item Voltaje de operación: 3.6 V - 6 V DC.
	\item Consumo corriente: 50 mA.
	\item Bluetooth: V2.0+EDR.
	\item Frecuencia: Banda ISM 2.4 GHz.
	\item Modulación: GFSK(Gaussian Frequency Shift Keying)
	\item Potencia de transmisión: 4 dBm, Class 2.
	\item Sensibilidad: -84dBm a 0.1% BER.
	\item Alcance 10 metros.
	\item Tamaño: 3,7 cm x 1,6 cmm.
\end{itemize}




%----------------------------------------------------------------------------------------
%	SECTION 2
%-----------------------------------------------------------------
\section{Requerimientos}

En esta sección se enumeran los requerimientos planteados en la planificación de trabajo final,elaborada al inicio del proyecto. Se encuentra dividida en requerimientos funcionales y requerimientos no funcionales.

\label{sec:requerimientos}

\begin{enumerate}
\item Requerimientos funcionales
	\begin{enumerate}
	\item Capacidad de locomoción.  El robot debe ser capaz de desplazarse por medio de ruedas motorizadas, a través de superficies planas.
	\item Capacidad de percepción. El robot debe ser capaz de detectar y obtener información del medio. 
	\item Capacidad de comunicación inalámbrica.
	\item El robot deberá funcionar con alimentación a batería recargable.
	\item El proyecto debe ser extensible a una posible herramienta de enseñanza e investigación

	\end{enumerate}
\item Requerimientos no funcionales
	\begin{enumerate}
	\item El robot no debe resultar peligroso para el ambiente o las personas con las que podría interactuar.
	\item El diseño del robot debe respetar regulaciones en cuanto a radiación en el espectro ultravioleta.
	\item Se utilizarán componentes electrónicos disponibles comercialmente en Argentina.
	\end{enumerate}
\end{enumerate}

\section{Planificación}

El trabajo se organizó para ser terminado en el mes de junio de 2021 con una dedicación aproximada de 600 horas en total. Con el fin de organizar y dar seguimiento a las actividades requeridas y poder identificar los desvíos en los tiempos de ejecución programados, se cuantificaron los tiempos de las diversas tareas mediante el diagrama de Gantt, que se observa en las figuras \ref{fig:gantt1} y \ref{fig:gantt2}


\begin{figure}[htpb]
\centering 
\includegraphics[width=\textwidth]{./Figures/gantttabla.PNG}
\caption{Tabla de tareas de Gantt}
\label{fig:gantt1}
\end{figure}

\begin{figure}[htpb]
\centering 
\includegraphics[width=.8\textwidth]{./Figures/Gantt.PNG}
\caption{Diagrama de Gantt}
\label{fig:gantt2}
\end{figure}

\pagebreak

Se confeccionó también un diagrama de Activity on Node, con la finalidad de resaltar las tareas cuyos retrasos podrían resultar críticos para la concreción del trabajo. En rojo se indica el camino crítico, como puede apreciarse en la figura \ref{fig:AoN}

\begin{figure}[htpb]
\centering 
\includegraphics[width=.8\textwidth]{./Figures/AoN.png}
\caption{Diagrama en \textit{Activity on Node}}
\label{fig:AoN}
\end{figure}


A partir de este análisis está organizado el trabajo que se presenta en los próximos capítulos.

 
	\chapter{Diseño e implementación} % Main chapter title

\label{Chapter3} % Change X to a consecutive number; for referencing this chapter elsewhere, use \ref{ChapterX}

\definecolor{mygreen}{rgb}{0,0.6,0}
\definecolor{mygray}{rgb}{0.5,0.5,0.5}
\definecolor{mymauve}{rgb}{0.58,0,0.82}

%%%%%%%%%%%%%%%%%%%%%%%%%%%%%%%%%%%%%%%%%%%%%%%%%%%%%%%%%%%%%%%%%%%%%%%%%%%%%
% parámetros para configurar el formato del código en los entornos lstlisting
%%%%%%%%%%%%%%%%%%%%%%%%%%%%%%%%%%%%%%%%%%%%%%%%%%%%%%%%%%%%%%%%%%%%%%%%%%%%%
\lstset{ %
  backgroundcolor=\color{white},   % choose the background color; you must add \usepackage{color} or \usepackage{xcolor}
  basicstyle=\footnotesize,        % the size of the fonts that are used for the code
  breakatwhitespace=false,         % sets if automatic breaks should only happen at whitespace
  breaklines=true,                 % sets automatic line breaking
  captionpos=b,                    % sets the caption-position to bottom
  commentstyle=\color{mygreen},    % comment style
  deletekeywords={...},            % if you want to delete keywords from the given language
  %escapeinside={\%*}{*)},          % if you want to add LaTeX within your code
  %extendedchars=true,              % lets you use non-ASCII characters; for 8-bits encodings only, does not work with UTF-8
  %frame=single,	                % adds a frame around the code
  keepspaces=true,                 % keeps spaces in text, useful for keeping indentation of code (possibly needs columns=flexible)
  keywordstyle=\color{blue},       % keyword style
  language=[ANSI]C,                % the language of the code
  %otherkeywords={*,...},           % if you want to add more keywords to the set
  numbers=left,                    % where to put the line-numbers; possible values are (none, left, right)
  numbersep=5pt,                   % how far the line-numbers are from the code
  numberstyle=\tiny\color{mygray}, % the style that is used for the line-numbers
  rulecolor=\color{black},         % if not set, the frame-color may be changed on line-breaks within not-black text (e.g. comments (green here))
  showspaces=false,                % show spaces everywhere adding particular underscores; it overrides 'showstringspaces'
  showstringspaces=false,          % underline spaces within strings only
  showtabs=false,                  % show tabs within strings adding particular underscores
  stepnumber=1,                    % the step between two line-numbers. If it's 1, each line will be numbered
  stringstyle=\color{mymauve},     % string literal style
  tabsize=2,	                   % sets default tabsize to 2 spaces
  title=\lstname,                  % show the filename of files included with \lstinputlisting; also try caption instead of title
  morecomment=[s]{/*}{*/}
}
En este capítulo se enumeran y desarrollan los aspectos considerados a la hora diseñar el robot. Se tuvieron en cuenta los alcances establecidos como así también las posibilidades económicas de solventar el proyecto.
%----------------------------------------------------------------------------------------
%	SECTION 1
%----------------------------------------------------------------------------------------
\section{Diseño de hardware}

En esta sección se detallan los componentes y módulos electrónicos que forman parte del robot. Se detalla la función que desempeña cada uno de ellos. En la figura \ref{fig:diagramaini} se puede apreciar un diagrama en bloques de los módulos que conforman el robot.


\begin{figure}[h]
	\centering
	\includegraphics[width=12cm]{./Figures/diagini.PNG}
	\caption{Diagrama en bloques del robot.}
	\label{fig:diagramaini}
\end{figure}



	\subsection{Poncho}
Se denominación  “Poncho”  se utiliza entre la comunidad del proyecto CIAA para referirse a una  placa de expansión de “shield”  que se conecta sobre algún procesador de la familia CIAA.  Para este proyecto se diseñó un poncho para facilitar las conexiones de la placa EDU-CIAA con los sensores, actuadores y el módulo de comunicación bluetooth.

		\subsubsection{Diseño esquemático del poncho}

El poncho consta de conectores para las señales de entrada:

\begin{itemize}
	\item Sensores infrarrojos (2).
	\item Finales de carrera (2).
	\item Sensor de movimiento (1).
\end{itemize}

A su vez, permite la conexión con los dispositivos de salida

\begin{itemize}
	\item Módulo de control de motores.
	\item Relé actuador (en la placa).
	\item Buzzer y LEDs (en la placa).
\end{itemize}

La placa posee conexionado para montar un módulo HC-05 de comunicación bluetooth y un conector destinado a dispositivos I2C (como podría ser un módulo de giróscopo o acelerómetro).

		\subsubsection{Diseño PCB del poncho}


La placa fue diseñada durante la cursada de la asignatura “diseño de circuitos impresos", según los lineamientos expuestos en la documentación para ponchos CIAA. Se procedió a la fabricación de la placa utilizando medios caseros de manufactura.

El diseño de PCB se realizó con el software KiCad \citep{KiCad} (Versión 5.1.9), el cual es un paquete de software para el diseño de circuitos electrónicos o EDA (Electronic Design
Automation). En la figura \ref{fig:poncho3d} se observa el modelo 3D de la placa y sus componentes.


\begin{figure}[h]
	\centering
	\includegraphics[width=11cm]{./Figures/ponchoiso.PNG}
	\caption{Vista del modelo 3D del poncho rUVot.}
	\label{fig:poncho3d}
\end{figure}


En la figura \ref{fig:esquematico} se presenta el circuito esquemático del poncho donde se puede observar el conexionado eléctrico.

%\begin{figure}[h]
%	\centering
%	\includegraphics[width=\textwidth]{./Figures/esquematico.png}
%	\caption{Circuito esquemático del poncho.}
%	\label{fig:esquematico}
%\end{figure}
%\pagebreak

\subsection{Esquema de  de comunicaciones}

\section{Diseño mecánico}
\subsection{Gabinete del robot}
\subsection{Motores}

\section{Diseño de software}
\subsection{Tarea de control de motores}
\subsection{Tarea de comunicaciones}







%
%
%
%
%\begin{verbatim}
%\begin{lstlisting}[caption= "un epígrafe descriptivo"]
%	las líneas de código irían aquí...
%\end{lstlisting}
%\end{verbatim}
%
%A modo de ejemplo:
%
%\begin{lstlisting}[label=cod:vControl,caption=Pseudocódigo del lazo principal de control.]  % Start your code-block
%
%#define MAX_SENSOR_NUMBER 3
%#define MAX_ALARM_NUMBER  6
%#define MAX_ACTUATOR_NUMBER 6
%
%uint32_t sensorValue[MAX_SENSOR_NUMBER];		
%FunctionalState alarmControl[MAX_ALARM_NUMBER];	//ENABLE or DISABLE
%state_t alarmState[MAX_ALARM_NUMBER];						//ON or OFF
%state_t actuatorState[MAX_ACTUATOR_NUMBER];			//ON or OFF
%
%void vControl() {
%
%	initGlobalVariables();
%	
%	period = 500 ms;
%		
%	while(1) {
%
%		ticks = xTaskGetTickCount();
%		
%		updateSensors();
%		
%		updateAlarms();
%		
%		controlActuators();
%		
%		vTaskDelayUntil(&ticks, period);
%	}
%}
%\end{lstlisting}




	% Chapter Template

\chapter{Ensayos y resultados} % Main chapter title

\label{Chapter4} % Change X to a consecutive number; for referencing this chapter elsewhere, use \ref{ChapterX}

%----------------------------------------------------------------------------------------
%	SECTION 1
%----------------------------------------------------------------------------------------

\section{Pruebas funcionales del hardware}
\label{sec:pruebasHW}

En este capítulo se detallan los ensayos realizados para comprobar el correcto funcionamiento de hardware y firmware, y la interacción de los módulos que componen el robot.

\subsection{Validación de movimientos del robot}
\subsection{Validación módulo de comunicaciones Bluetooth}
\subsection{Validación detección de obstáculos}
\subsection{Validación de navegación autónoma}


\section{Pruebas no Funcionales}
%\subsection{Tarea de comunicaciones} 
	% Chapter Template

\chapter{Conclusiones} % Main chapter title

\label{Chapter5} % Change X to a consecutive number; for referencing this chapter elsewhere, use \ref{ChapterX}


%----------------------------------------------------------------------------------------

%----------------------------------------------------------------------------------------
%	SECTION 1
%----------------------------------------------------------------------------------------


En este capítulo se presenta un breve resumen del trabajo realizado, los problemas encontrados y los resultados obtenidos. También se mencionan mejoras a realizar en el futuro.

\section{Resultados obtenidos}

El trabajo finalizó con el desarrollo de un prototipo de robot móvil para tareas de desinfección por efecto de rayos ultravioletas germicidas. Se cumplieron los requerimientos planteados en la planificación del trabajo.
Se desarrolló con éxito un circuito impreso como placa de expansión de hardware, y un firmware funcional para la placa EDU-CIAA  
Se verificó el funcionamiento en el modo autónomo, en el que realiza un recorrido evitando obstáculos, como así también  en el  modo de teleoperación, en el que puede controlarse a distancia desde una  aplicación en un celular o Tablet.
El dispositivo puede ser usado para desinfección sin residuos químicos en espacios públicos y en el hogar. 

La planificación, se cumplió dentro de los plazos esperados, aunque se manifestó el riesgo “Imposibilidad de cumplir con los plazos planteados para el desarrollo del proyecto”. Esto se debió a la  reducción de tiempo disponible para dedicarlo al proyecto,  debido a actividades laborales y estar cursando las últimas materias de le carrera.  Al haber extendido el plazo para la entrega y haber re-planificado actividades se logró mitigar este inconveniente.


%----------------------------------------------------------------------------------------
%	SECTION 2
%----------------------------------------------------------------------------------------
\section{Conocimientos aplicados}

Durante la realización de este trabajo se aplicaron conocimientos adquiridos en el transcurso de la carrera de especialización. 
En particular, fueron importantes los aportes de las siguiente  asignaturas:


\begin{itemize}
	\item Gestión de proyectos, para realizar la planificación y generar toda la documentación inicial.
	\item Ingeniería de software para definir los requerimientos básicos y pensar el proyecto desde las necesidades del usuario.También se aplicaron los conocimientos relativos a la implementación de un repositorio GIT para el resguardo y versionado de toda la documentación del proyecto.
	\item Programación de microcontroladores para la implementación del firmware en C del microcontrolador ARM Cortex-M4 de la placa EDU-CIAA. En la asignatura se presentó todo lo referente a la modularización por archivos implementada en este trabajo y el modelo de máquinas de estado finito.
	\item Protocolos de comunicaciones en sistemas embebidos, para conocer las posibilidades de comunicación de la placa EDU-CIAA con otros dispositivos, en particular con el módulo Bluetooth. 
	\item Diseño de Circuitos Impresos, para el desarrollo de la placa de expansión de hardware (poncho) utilizada en este trabajo, y el aprendizaje de buenas costumbres de diseño de PCB
			
\end{itemize}

%----------------------------------------------------------------------------------------
%	SECTION 3
%----------------------------------------------------------------------------------------
\section{Próximos pasos}

Como mejoras a futuro se contempla:

\begin{itemize}
	\item Agregar una unidad de medición inercial o IMU (por su sigla en inglés) como ser un acelerómetro o un giróscopo, para tener informa acerca de la velocidad y orientación del robot  en el modo autónomo. De esta manera se podría ampliar la variedad de recorridos posibles y que no dependan únicamente de las características del entorno. 
	\item Al contar con un puerto I2C en la placa, sería posible incorporar un lector de tarjetas de memoria (tipo SD) para almacenar allí la librería con la que se configura la máquina de estados principal. Con este aditamento sería posible definir o ampliar el comportamiento autónomo del robot sin necesidad de modificar su programación. 		
	\item Ya que la placa de expansión de hardware utiliza un relé para conmutar el módulo UVC, podían desarrollarse otros módulos (intercambiables) con su propia alimentación, que utilicen diferentes lámparas germicidas o que ofrezcan otras prestaciones.  
\end{itemize} 
\end{verbatim}

Los apéndices también deben escribirse en archivos .tex separados, que se deben ubicar dentro de la carpeta \emph{Appendices}. Los apéndices vienen comentados por defecto con el caracter \code{\%} y para incluirlos simplemente se debe eliminar dicho caracter.

Finalmente, se encuentra el código para incluir la bibliografía en el documento final.  Este código tampoco debe modificarse. La metodología para trabajar las referencias bibliográficas se desarrolla en la sección \ref{sec:biblio}.
%----------------------------------------------------------------------------------------

\section{Bibliografía}
\label{sec:biblio}

Las opciones de formato de la bibliografía se controlan a través del paquete de latex \option{biblatex} que se incluye en la memoria en el archivo memoria.tex.  Estas opciones determinan cómo se generan las citas bibliográficas en el cuerpo del documento y cómo se genera la bibliografía al final de la memoria.

En el preámbulo se puede encontrar el código que incluye el paquete biblatex, que no requiere ninguna modificación del usuario de la plantilla, y que contiene las siguientes opciones:

\begin{lstlisting}
\usepackage[backend=bibtex,
	natbib=true, 
	style=numeric, 
	sorting=none]
{biblatex}
\end{lstlisting}

En el archivo \file{reference.bib} se encuentran las referencias bibliográficas que se pueden citar en el documento.  Para incorporar una nueva cita al documento lo primero es agregarla en este archivo con todos los campos necesario.  Todas las entradas bibliográficas comienzan con $@$ y una palabra que define el formato de la entrada.  Para cada formato existen campos obligatorios que deben completarse. No importa el orden en que las entradas estén definidas en el archivo .bib.  Tampoco es importante el orden en que estén definidos los campos de una entrada bibliográfica. A continuación se muestran algunos ejemplos:

\begin{lstlisting}
@ARTICLE{ARTICLE:1,
    AUTHOR="John Doe",
    TITLE="Title",
    JOURNAL="Journal",
    YEAR="2017",
}
\end{lstlisting}


\begin{lstlisting}
@BOOK{BOOK:1,
    AUTHOR="John Doe",
    TITLE="The Book without Title",
    PUBLISHER="Dummy Publisher",
    YEAR="2100",
}
\end{lstlisting}


\begin{lstlisting}
@INBOOK{BOOK:2,
    AUTHOR="John Doe",
    TITLE="The Book without Title",
    PUBLISHER="Dummy Publisher",
    YEAR="2100",
    PAGES="100-200",
}
\end{lstlisting}


\begin{lstlisting}
@MISC{WEBSITE:1,
    HOWPUBLISHED = "\url{http://example.com}",
    AUTHOR = "Intel",
    TITLE = "Example Website",
    MONTH = "12",
    YEAR = "1988",
    URLDATE = {2012-11-26}
}
\end{lstlisting}

Se debe notar que los nombres \emph{ARTICLE:1}, \emph{BOOK:1}, \emph{BOOK:2} y \emph{WEBSITE:1} son nombres de fantasía que le sirve al autor del documento para identificar la entrada. En este sentido, se podrían reemplazar por cualquier otro nombre.  Tampoco es necesario poner : seguido de un número, en los ejemplos sólo se incluye como un posible estilo para identificar las entradas.

La entradas se citan en el documento con el comando: 

\begin{verbatim}
\citep{nombre_de_la_entrada}
\end{verbatim}

Y cuando se usan, se muestran así: \citep{ARTICLE:1}, \citep{BOOK:1}, \citep{BOOK:2}, \citep{WEBSITE:1}.  Notar cómo se conforma la sección Bibliografía al final del documento. 

	\chapter{Introducción específica} % Main chapter title

\label{Chapter2}

%----------------------------------------------------------------------------------------
%	SECTION 1
%----------------------------------------------------------------------------------------
En este capítulo se presentan las distintas tecnologías y metodologías disponibles para la implementación del prototipo de robot móvil. Se describen los dispositivos y arquitecturas más significativos que permitieron alcanzar los requerimientos planteados.

\section{Criterios de diseño del robot}

En esta sección se enumeran los aspectos considerados a la hora diseñar el robot. Se tuvieron en cuenta los alcances establecidos como así también las posibilidades económicas de solventar el proyecto. 



\subsection{Placa de microprocesamiento}
Se utilizó  la placa de desarrollo EDUCIAA-NXP \citep{EDUCIAA} ya que la misma se usa para la ejercitación en varias asignaturas de la carrera de postgrado. En la figura \ref{fig:EDUCIAANXP} se observa una imagen de la EDU-CIAA-NXP, que es una versión de bajo costo de la CIAA-NXP, pensada para la enseñanza universitaria, terciaria y secundaria. 

\begin{figure}[htpb]
	\centering
	\includegraphics[width=\textwidth]{./Figures/EDUCIAANXP.jpg}
	\caption{placa de desarrollo EDUCIAA-NXP.\protect\footnotemark.}
	\label{fig:EDUCIAANXP}
\end{figure}
\footnotetext{Imagen tomada de \url{http://www.proyecto-ciaa.com.ar}}



\begin{figure}[htpb]
	\centering
	\includegraphics[width=9cm]{./Figures/Bloques.jpg}
	\caption{Diagrama en bloques de la EDUCIAA-NXP.\protect\footnotemark.}
	\label{fig:Bloques}
\end{figure}
\footnotetext{Imagen tomada de \url{http://www.proyecto-ciaa.com.ar}}


En la figura \ref{fig:Bloques} puede verse un diagrama en bloques general de la placa. 

El  microcontrolador utilizado por la EDU-CIAA es el LPC4337 (dual core ARM Cortex-M4F y Cortex-M0). Los recursos más significativos que se utilizaron de la placa fueron: 


\begin{itemize}
	\item GPIO (General Purpose Input/Output, Entrada/Salida de Propósito General)
	\item PWM (Pulse Width Modulation, modulación por ancho de pulso).
	\item UART (Universal Asynchronous Receiver-Transmitter, Transmisor-Receptor Asíncrono Universal).
	\item Temporizadores.
\end{itemize}



\subsection{Driver de motores}

Se utilizó un módulo para el accionamiento de motores \citep{Driver}. En la figura \ref{fig:Driver} se puede observar una imagen de la placa 

\begin{figure}[h]
	\centering
	\includegraphics[width=5cm]{./Figures/L298N.png}
	\caption{Driver de motores.\protect\footnotemark.}
	\label{fig:Driver}
\end{figure}
\footnotetext{Imagen tomada de \url{http://robots-argentina.com.ar}}


El módulo está basado en el circuito integrado L298N \citep{L298} y permite controlar dos motores de corriente continua de manera simultánea e independiente.  Sus características principales son:

\begin{itemize}
	\item Tensión mínima: 5 V.
	\item Tensión máxima: 35 V.
	\item Corriente máxima: 2 A.
	\item Tensión de nivel lógico: 5 V.
	\item Potencia máxima 25 W
	\item Medidas: 43 x 43 x 24 mm
\end{itemize}

La placa tiene la opción de habilitar o no el regulador LM7805 integrado para alimentar la parte lógica. En la figura \ref{fig:Esquema} se observa el diagrama esquemático del módulo.


\begin{figure}[h]
	\centering
	\includegraphics[width=12cm]{./Figures/Modulo.png}
	\caption{diagrama esquemático del módulo.\protect\footnotemark.}
	\label{fig:Esquema}
\end{figure}
\footnotetext{Imagen tomada de \url{http://robots-argentina.com.ar}}
\pagebreak



\subsection{Módulo sensor de Infrarrojos}

Se utilizaron dos módulos sensores de proximidad por infrarrojos IR FC-51 \citep{IR} para la detección de obstáculos por parte del robot. Estos módulos están compuestos por  un emisor de luz infrarroja (IR)  y un receptor que detecta su reflejo en  las superficies contra las que se enfrenta, de modo que presentan una señal en  presencia de cualquier obstáculo en su parte frontal. Un potenciómetro permite al ajustar el rango de detección. 

El sensor presenta una respuesta estable incluso con luz ambiente o en completa oscuridad. En la figura \ref{fig:moduloIR} se observa una imagen del sensor de Infrarrojos.

\begin{figure}[h]
	\centering
	\includegraphics[width=8cm]{./Figures/moduloIR.jpg}
	\caption{Módulo sensor de Infrarrojos .\protect\footnotemark.}
	\label{fig:moduloIR}
\end{figure}
\footnotetext{Imagen tomada de \url{http://robots-argentina.com.ar}}



En la figura \ref{fig:IRschem} se muestra el circuito esquemático  del sensor de Infrarrojos

\begin{figure}[h]
	\centering
	\includegraphics[width=14cm]{./Figures/IRschem.jpg}
	\caption{Esquema del módulo sensor de Infrarrojos .\protect\footnotemark.}
	\label{fig:IRschem}
\end{figure}
\footnotetext{Imagen tomada de \url{http://robots-argentina.com.ar}}
\pagebreak

Las características del módulo son:
\begin{itemize}
	\item Ángulo de cobertura: 35°.
	\item Tensión de funcionamiento: 3 V – 6 V.
	\item Rango de detección: 2 cm – 30 cm (ajustable con el potenciómetro).
	\item Tamaño: 4,5 cm x 1,4 cm x 0.7cm. 
	\item Discriminación: Las salida toma nivel lógico bajo cuando se detecta un obstáculo (reflexión).
\end{itemize}



\subsection{Baterías}

En función del consumo y la intensión de no dedicar mayor espacio a las celdas de alimentación, se emplearon dos baterías de Ion-litio tipo 18650. Una de las ventajas de las Ion-litio es que permiten ser recargadas con una media de entre 600 a 1000 veces sin que se estropeen ni pierdan efectividad \citep{18650}. La capacidad de estas baterías varían de un modelo a otro pero suelen estar comprendidas entre los 2100 maH y los 4000maH.Su tensión nominal es de 3,7 V (hasta 4,2 V en vacío).

Las baterías van conectadas en serie para lograr una tensión de 7,4 V acorde a la alimentación de los motores y con un margen superior necesario para el correcto funcionamiento del regulador de tensión de 5 V del módulo de accionamiento de motores. 
Las baterías se insertaron en un portapilas comercial. En la figura \ref{fig:portapila} se muestra las dos baterías 18650 ya instaladas en su portapila.


\begin{figure}[h]
	\centering
	\includegraphics[width=10cm]{./Figures/portapilas.PNG}
	\caption{Las dos baterías 18650 en su portapila .}
	\label{fig:portapila}
\end{figure}


\subsection{Módulo de comunicaciones Bluetooth}

Se utilizó el módulo Bluetooth HC-05 para la comunicación comunicaciones con el robot \citep{HC05}. El mismo ya había sido utilizado en  prácticas de la asignatura “Protocolos de Comunicaciones en sistemas embebidos”, conectado al puerto serie de la EDU-CIAA. Todos los parámetros del módulo se pueden configurar mediante comandos AT. 

En la figura \ref{fig:moduloHC05} se observa el módulo HC-05.


\begin{figure}[h]
	\centering
	\includegraphics[width=8cm]{./Figures/HC05.jpeg}
	\caption{Módulo Bluetooth HC-05 .\protect\footnotemark.}
	\label{fig:moduloHC05}
\end{figure}
\footnotetext{Imagen tomada de \url{https://maker.pro/custom/tutorial/hc-05-bluetooth-transceiver-module-datasheet-highlights}}



Las características del módulo son:
\begin{itemize}
	\item Voltaje de operación: 3.6 V - 6 V DC.
	\item Consumo corriente: 50 mA.
	\item Bluetooth: V2.0+EDR.
	\item Frecuencia: Banda ISM 2.4 GHz.
	\item Modulación: GFSK(Gaussian Frequency Shift Keying)
	\item Potencia de transmisión: 4 dBm, Class 2.
	\item Sensibilidad: -84dBm a 0.1% BER.
	\item Alcance 10 metros.
	\item Tamaño: 3,7 cm x 1,6 cmm.
\end{itemize}




%----------------------------------------------------------------------------------------
%	SECTION 2
%-----------------------------------------------------------------
\section{Requerimientos}

En esta sección se enumeran los requerimientos planteados en la planificación de trabajo final,elaborada al inicio del proyecto. Se encuentra dividida en requerimientos funcionales y requerimientos no funcionales.

\label{sec:requerimientos}

\begin{enumerate}
\item Requerimientos funcionales
	\begin{enumerate}
	\item Capacidad de locomoción.  El robot debe ser capaz de desplazarse por medio de ruedas motorizadas, a través de superficies planas.
	\item Capacidad de percepción. El robot debe ser capaz de detectar y obtener información del medio. 
	\item Capacidad de comunicación inalámbrica.
	\item El robot deberá funcionar con alimentación a batería recargable.
	\item El proyecto debe ser extensible a una posible herramienta de enseñanza e investigación

	\end{enumerate}
\item Requerimientos no funcionales
	\begin{enumerate}
	\item El robot no debe resultar peligroso para el ambiente o las personas con las que podría interactuar.
	\item El diseño del robot debe respetar regulaciones en cuanto a radiación en el espectro ultravioleta.
	\item Se utilizarán componentes electrónicos disponibles comercialmente en Argentina.
	\end{enumerate}
\end{enumerate}

\section{Planificación}

El trabajo se organizó para ser terminado en el mes de junio de 2021 con una dedicación aproximada de 600 horas en total. Con el fin de organizar y dar seguimiento a las actividades requeridas y poder identificar los desvíos en los tiempos de ejecución programados, se cuantificaron los tiempos de las diversas tareas mediante el diagrama de Gantt, que se observa en las figuras \ref{fig:gantt1} y \ref{fig:gantt2}


\begin{figure}[htpb]
\centering 
\includegraphics[width=\textwidth]{./Figures/gantttabla.PNG}
\caption{Tabla de tareas de Gantt}
\label{fig:gantt1}
\end{figure}

\begin{figure}[htpb]
\centering 
\includegraphics[width=.8\textwidth]{./Figures/Gantt.PNG}
\caption{Diagrama de Gantt}
\label{fig:gantt2}
\end{figure}

\pagebreak

Se confeccionó también un diagrama de Activity on Node, con la finalidad de resaltar las tareas cuyos retrasos podrían resultar críticos para la concreción del trabajo. En rojo se indica el camino crítico, como puede apreciarse en la figura \ref{fig:AoN}

\begin{figure}[htpb]
\centering 
\includegraphics[width=.8\textwidth]{./Figures/AoN.png}
\caption{Diagrama en \textit{Activity on Node}}
\label{fig:AoN}
\end{figure}


A partir de este análisis está organizado el trabajo que se presenta en los próximos capítulos.

 
	\chapter{Diseño e implementación} % Main chapter title

\label{Chapter3} % Change X to a consecutive number; for referencing this chapter elsewhere, use \ref{ChapterX}

\definecolor{mygreen}{rgb}{0,0.6,0}
\definecolor{mygray}{rgb}{0.5,0.5,0.5}
\definecolor{mymauve}{rgb}{0.58,0,0.82}

%%%%%%%%%%%%%%%%%%%%%%%%%%%%%%%%%%%%%%%%%%%%%%%%%%%%%%%%%%%%%%%%%%%%%%%%%%%%%
% parámetros para configurar el formato del código en los entornos lstlisting
%%%%%%%%%%%%%%%%%%%%%%%%%%%%%%%%%%%%%%%%%%%%%%%%%%%%%%%%%%%%%%%%%%%%%%%%%%%%%
\lstset{ %
  backgroundcolor=\color{white},   % choose the background color; you must add \usepackage{color} or \usepackage{xcolor}
  basicstyle=\footnotesize,        % the size of the fonts that are used for the code
  breakatwhitespace=false,         % sets if automatic breaks should only happen at whitespace
  breaklines=true,                 % sets automatic line breaking
  captionpos=b,                    % sets the caption-position to bottom
  commentstyle=\color{mygreen},    % comment style
  deletekeywords={...},            % if you want to delete keywords from the given language
  %escapeinside={\%*}{*)},          % if you want to add LaTeX within your code
  %extendedchars=true,              % lets you use non-ASCII characters; for 8-bits encodings only, does not work with UTF-8
  %frame=single,	                % adds a frame around the code
  keepspaces=true,                 % keeps spaces in text, useful for keeping indentation of code (possibly needs columns=flexible)
  keywordstyle=\color{blue},       % keyword style
  language=[ANSI]C,                % the language of the code
  %otherkeywords={*,...},           % if you want to add more keywords to the set
  numbers=left,                    % where to put the line-numbers; possible values are (none, left, right)
  numbersep=5pt,                   % how far the line-numbers are from the code
  numberstyle=\tiny\color{mygray}, % the style that is used for the line-numbers
  rulecolor=\color{black},         % if not set, the frame-color may be changed on line-breaks within not-black text (e.g. comments (green here))
  showspaces=false,                % show spaces everywhere adding particular underscores; it overrides 'showstringspaces'
  showstringspaces=false,          % underline spaces within strings only
  showtabs=false,                  % show tabs within strings adding particular underscores
  stepnumber=1,                    % the step between two line-numbers. If it's 1, each line will be numbered
  stringstyle=\color{mymauve},     % string literal style
  tabsize=2,	                   % sets default tabsize to 2 spaces
  title=\lstname,                  % show the filename of files included with \lstinputlisting; also try caption instead of title
  morecomment=[s]{/*}{*/}
}
En este capítulo se enumeran y desarrollan los aspectos considerados a la hora diseñar el robot. Se tuvieron en cuenta los alcances establecidos como así también las posibilidades económicas de solventar el proyecto.
%----------------------------------------------------------------------------------------
%	SECTION 1
%----------------------------------------------------------------------------------------
\section{Diseño de hardware}

En esta sección se detallan los componentes y módulos electrónicos que forman parte del robot. Se detalla la función que desempeña cada uno de ellos. En la figura \ref{fig:diagramaini} se puede apreciar un diagrama en bloques de los módulos que conforman el robot.


\begin{figure}[h]
	\centering
	\includegraphics[width=12cm]{./Figures/diagini.PNG}
	\caption{Diagrama en bloques del robot.}
	\label{fig:diagramaini}
\end{figure}



	\subsection{Poncho}
Se denominación  “Poncho”  se utiliza entre la comunidad del proyecto CIAA para referirse a una  placa de expansión de “shield”  que se conecta sobre algún procesador de la familia CIAA.  Para este proyecto se diseñó un poncho para facilitar las conexiones de la placa EDU-CIAA con los sensores, actuadores y el módulo de comunicación bluetooth.

		\subsubsection{Diseño esquemático del poncho}

El poncho consta de conectores para las señales de entrada:

\begin{itemize}
	\item Sensores infrarrojos (2).
	\item Finales de carrera (2).
	\item Sensor de movimiento (1).
\end{itemize}

A su vez, permite la conexión con los dispositivos de salida

\begin{itemize}
	\item Módulo de control de motores.
	\item Relé actuador (en la placa).
	\item Buzzer y LEDs (en la placa).
\end{itemize}

La placa posee conexionado para montar un módulo HC-05 de comunicación bluetooth y un conector destinado a dispositivos I2C (como podría ser un módulo de giróscopo o acelerómetro).

		\subsubsection{Diseño PCB del poncho}


La placa fue diseñada durante la cursada de la asignatura “diseño de circuitos impresos", según los lineamientos expuestos en la documentación para ponchos CIAA. Se procedió a la fabricación de la placa utilizando medios caseros de manufactura.

El diseño de PCB se realizó con el software KiCad \citep{KiCad} (Versión 5.1.9), el cual es un paquete de software para el diseño de circuitos electrónicos o EDA (Electronic Design
Automation). En la figura \ref{fig:poncho3d} se observa el modelo 3D de la placa y sus componentes.


\begin{figure}[h]
	\centering
	\includegraphics[width=11cm]{./Figures/ponchoiso.PNG}
	\caption{Vista del modelo 3D del poncho rUVot.}
	\label{fig:poncho3d}
\end{figure}


En la figura \ref{fig:esquematico} se presenta el circuito esquemático del poncho donde se puede observar el conexionado eléctrico.

%\begin{figure}[h]
%	\centering
%	\includegraphics[width=\textwidth]{./Figures/esquematico.png}
%	\caption{Circuito esquemático del poncho.}
%	\label{fig:esquematico}
%\end{figure}
%\pagebreak

\subsection{Esquema de  de comunicaciones}

\section{Diseño mecánico}
\subsection{Gabinete del robot}
\subsection{Motores}

\section{Diseño de software}
\subsection{Tarea de control de motores}
\subsection{Tarea de comunicaciones}







%
%
%
%
%\begin{verbatim}
%\begin{lstlisting}[caption= "un epígrafe descriptivo"]
%	las líneas de código irían aquí...
%\end{lstlisting}
%\end{verbatim}
%
%A modo de ejemplo:
%
%\begin{lstlisting}[label=cod:vControl,caption=Pseudocódigo del lazo principal de control.]  % Start your code-block
%
%#define MAX_SENSOR_NUMBER 3
%#define MAX_ALARM_NUMBER  6
%#define MAX_ACTUATOR_NUMBER 6
%
%uint32_t sensorValue[MAX_SENSOR_NUMBER];		
%FunctionalState alarmControl[MAX_ALARM_NUMBER];	//ENABLE or DISABLE
%state_t alarmState[MAX_ALARM_NUMBER];						//ON or OFF
%state_t actuatorState[MAX_ACTUATOR_NUMBER];			//ON or OFF
%
%void vControl() {
%
%	initGlobalVariables();
%	
%	period = 500 ms;
%		
%	while(1) {
%
%		ticks = xTaskGetTickCount();
%		
%		updateSensors();
%		
%		updateAlarms();
%		
%		controlActuators();
%		
%		vTaskDelayUntil(&ticks, period);
%	}
%}
%\end{lstlisting}




	% Chapter Template

\chapter{Ensayos y resultados} % Main chapter title

\label{Chapter4} % Change X to a consecutive number; for referencing this chapter elsewhere, use \ref{ChapterX}

%----------------------------------------------------------------------------------------
%	SECTION 1
%----------------------------------------------------------------------------------------

\section{Pruebas funcionales del hardware}
\label{sec:pruebasHW}

En este capítulo se detallan los ensayos realizados para comprobar el correcto funcionamiento de hardware y firmware, y la interacción de los módulos que componen el robot.

\subsection{Validación de movimientos del robot}
\subsection{Validación módulo de comunicaciones Bluetooth}
\subsection{Validación detección de obstáculos}
\subsection{Validación de navegación autónoma}


\section{Pruebas no Funcionales}
%\subsection{Tarea de comunicaciones} 
	% Chapter Template

\chapter{Conclusiones} % Main chapter title

\label{Chapter5} % Change X to a consecutive number; for referencing this chapter elsewhere, use \ref{ChapterX}


%----------------------------------------------------------------------------------------

%----------------------------------------------------------------------------------------
%	SECTION 1
%----------------------------------------------------------------------------------------


En este capítulo se presenta un breve resumen del trabajo realizado, los problemas encontrados y los resultados obtenidos. También se mencionan mejoras a realizar en el futuro.

\section{Resultados obtenidos}

El trabajo finalizó con el desarrollo de un prototipo de robot móvil para tareas de desinfección por efecto de rayos ultravioletas germicidas. Se cumplieron los requerimientos planteados en la planificación del trabajo.
Se desarrolló con éxito un circuito impreso como placa de expansión de hardware, y un firmware funcional para la placa EDU-CIAA  
Se verificó el funcionamiento en el modo autónomo, en el que realiza un recorrido evitando obstáculos, como así también  en el  modo de teleoperación, en el que puede controlarse a distancia desde una  aplicación en un celular o Tablet.
El dispositivo puede ser usado para desinfección sin residuos químicos en espacios públicos y en el hogar. 

La planificación, se cumplió dentro de los plazos esperados, aunque se manifestó el riesgo “Imposibilidad de cumplir con los plazos planteados para el desarrollo del proyecto”. Esto se debió a la  reducción de tiempo disponible para dedicarlo al proyecto,  debido a actividades laborales y estar cursando las últimas materias de le carrera.  Al haber extendido el plazo para la entrega y haber re-planificado actividades se logró mitigar este inconveniente.


%----------------------------------------------------------------------------------------
%	SECTION 2
%----------------------------------------------------------------------------------------
\section{Conocimientos aplicados}

Durante la realización de este trabajo se aplicaron conocimientos adquiridos en el transcurso de la carrera de especialización. 
En particular, fueron importantes los aportes de las siguiente  asignaturas:


\begin{itemize}
	\item Gestión de proyectos, para realizar la planificación y generar toda la documentación inicial.
	\item Ingeniería de software para definir los requerimientos básicos y pensar el proyecto desde las necesidades del usuario.También se aplicaron los conocimientos relativos a la implementación de un repositorio GIT para el resguardo y versionado de toda la documentación del proyecto.
	\item Programación de microcontroladores para la implementación del firmware en C del microcontrolador ARM Cortex-M4 de la placa EDU-CIAA. En la asignatura se presentó todo lo referente a la modularización por archivos implementada en este trabajo y el modelo de máquinas de estado finito.
	\item Protocolos de comunicaciones en sistemas embebidos, para conocer las posibilidades de comunicación de la placa EDU-CIAA con otros dispositivos, en particular con el módulo Bluetooth. 
	\item Diseño de Circuitos Impresos, para el desarrollo de la placa de expansión de hardware (poncho) utilizada en este trabajo, y el aprendizaje de buenas costumbres de diseño de PCB
			
\end{itemize}

%----------------------------------------------------------------------------------------
%	SECTION 3
%----------------------------------------------------------------------------------------
\section{Próximos pasos}

Como mejoras a futuro se contempla:

\begin{itemize}
	\item Agregar una unidad de medición inercial o IMU (por su sigla en inglés) como ser un acelerómetro o un giróscopo, para tener informa acerca de la velocidad y orientación del robot  en el modo autónomo. De esta manera se podría ampliar la variedad de recorridos posibles y que no dependan únicamente de las características del entorno. 
	\item Al contar con un puerto I2C en la placa, sería posible incorporar un lector de tarjetas de memoria (tipo SD) para almacenar allí la librería con la que se configura la máquina de estados principal. Con este aditamento sería posible definir o ampliar el comportamiento autónomo del robot sin necesidad de modificar su programación. 		
	\item Ya que la placa de expansión de hardware utiliza un relé para conmutar el módulo UVC, podían desarrollarse otros módulos (intercambiables) con su propia alimentación, que utilicen diferentes lámparas germicidas o que ofrezcan otras prestaciones.  
\end{itemize} 
\end{verbatim}

Los apéndices también deben escribirse en archivos .tex separados, que se deben ubicar dentro de la carpeta \emph{Appendices}. Los apéndices vienen comentados por defecto con el caracter \code{\%} y para incluirlos simplemente se debe eliminar dicho caracter.

Finalmente, se encuentra el código para incluir la bibliografía en el documento final.  Este código tampoco debe modificarse. La metodología para trabajar las referencias bibliográficas se desarrolla en la sección \ref{sec:biblio}.
%----------------------------------------------------------------------------------------

\section{Bibliografía}
\label{sec:biblio}

Las opciones de formato de la bibliografía se controlan a través del paquete de latex \option{biblatex} que se incluye en la memoria en el archivo memoria.tex.  Estas opciones determinan cómo se generan las citas bibliográficas en el cuerpo del documento y cómo se genera la bibliografía al final de la memoria.

En el preámbulo se puede encontrar el código que incluye el paquete biblatex, que no requiere ninguna modificación del usuario de la plantilla, y que contiene las siguientes opciones:

\begin{lstlisting}
\usepackage[backend=bibtex,
	natbib=true, 
	style=numeric, 
	sorting=none]
{biblatex}
\end{lstlisting}

En el archivo \file{reference.bib} se encuentran las referencias bibliográficas que se pueden citar en el documento.  Para incorporar una nueva cita al documento lo primero es agregarla en este archivo con todos los campos necesario.  Todas las entradas bibliográficas comienzan con $@$ y una palabra que define el formato de la entrada.  Para cada formato existen campos obligatorios que deben completarse. No importa el orden en que las entradas estén definidas en el archivo .bib.  Tampoco es importante el orden en que estén definidos los campos de una entrada bibliográfica. A continuación se muestran algunos ejemplos:

\begin{lstlisting}
@ARTICLE{ARTICLE:1,
    AUTHOR="John Doe",
    TITLE="Title",
    JOURNAL="Journal",
    YEAR="2017",
}
\end{lstlisting}


\begin{lstlisting}
@BOOK{BOOK:1,
    AUTHOR="John Doe",
    TITLE="The Book without Title",
    PUBLISHER="Dummy Publisher",
    YEAR="2100",
}
\end{lstlisting}


\begin{lstlisting}
@INBOOK{BOOK:2,
    AUTHOR="John Doe",
    TITLE="The Book without Title",
    PUBLISHER="Dummy Publisher",
    YEAR="2100",
    PAGES="100-200",
}
\end{lstlisting}


\begin{lstlisting}
@MISC{WEBSITE:1,
    HOWPUBLISHED = "\url{http://example.com}",
    AUTHOR = "Intel",
    TITLE = "Example Website",
    MONTH = "12",
    YEAR = "1988",
    URLDATE = {2012-11-26}
}
\end{lstlisting}

Se debe notar que los nombres \emph{ARTICLE:1}, \emph{BOOK:1}, \emph{BOOK:2} y \emph{WEBSITE:1} son nombres de fantasía que le sirve al autor del documento para identificar la entrada. En este sentido, se podrían reemplazar por cualquier otro nombre.  Tampoco es necesario poner : seguido de un número, en los ejemplos sólo se incluye como un posible estilo para identificar las entradas.

La entradas se citan en el documento con el comando: 

\begin{verbatim}
\citep{nombre_de_la_entrada}
\end{verbatim}

Y cuando se usan, se muestran así: \citep{ARTICLE:1}, \citep{BOOK:1}, \citep{BOOK:2}, \citep{WEBSITE:1}.  Notar cómo se conforma la sección Bibliografía al final del documento. 

	\chapter{Introducción específica} % Main chapter title

\label{Chapter2}

%----------------------------------------------------------------------------------------
%	SECTION 1
%----------------------------------------------------------------------------------------
En este capítulo se presentan las distintas tecnologías y metodologías disponibles para la implementación del prototipo de robot móvil. Se describen los dispositivos y arquitecturas más significativos que permitieron alcanzar los requerimientos planteados.

\section{Criterios de diseño del robot}

En esta sección se enumeran los aspectos considerados a la hora diseñar el robot. Se tuvieron en cuenta los alcances establecidos como así también las posibilidades económicas de solventar el proyecto. 



\subsection{Placa de microprocesamiento}
Se utilizó  la placa de desarrollo EDUCIAA-NXP \citep{EDUCIAA} ya que la misma se usa para la ejercitación en varias asignaturas de la carrera de postgrado. En la figura \ref{fig:EDUCIAANXP} se observa una imagen de la EDU-CIAA-NXP, que es una versión de bajo costo de la CIAA-NXP, pensada para la enseñanza universitaria, terciaria y secundaria. 

\begin{figure}[htpb]
	\centering
	\includegraphics[width=\textwidth]{./Figures/EDUCIAANXP.jpg}
	\caption{placa de desarrollo EDUCIAA-NXP.\protect\footnotemark.}
	\label{fig:EDUCIAANXP}
\end{figure}
\footnotetext{Imagen tomada de \url{http://www.proyecto-ciaa.com.ar}}



\begin{figure}[htpb]
	\centering
	\includegraphics[width=9cm]{./Figures/Bloques.jpg}
	\caption{Diagrama en bloques de la EDUCIAA-NXP.\protect\footnotemark.}
	\label{fig:Bloques}
\end{figure}
\footnotetext{Imagen tomada de \url{http://www.proyecto-ciaa.com.ar}}


En la figura \ref{fig:Bloques} puede verse un diagrama en bloques general de la placa. 

El  microcontrolador utilizado por la EDU-CIAA es el LPC4337 (dual core ARM Cortex-M4F y Cortex-M0). Los recursos más significativos que se utilizaron de la placa fueron: 


\begin{itemize}
	\item GPIO (General Purpose Input/Output, Entrada/Salida de Propósito General)
	\item PWM (Pulse Width Modulation, modulación por ancho de pulso).
	\item UART (Universal Asynchronous Receiver-Transmitter, Transmisor-Receptor Asíncrono Universal).
	\item Temporizadores.
\end{itemize}



\subsection{Driver de motores}

Se utilizó un módulo para el accionamiento de motores \citep{Driver}. En la figura \ref{fig:Driver} se puede observar una imagen de la placa 

\begin{figure}[h]
	\centering
	\includegraphics[width=5cm]{./Figures/L298N.png}
	\caption{Driver de motores.\protect\footnotemark.}
	\label{fig:Driver}
\end{figure}
\footnotetext{Imagen tomada de \url{http://robots-argentina.com.ar}}


El módulo está basado en el circuito integrado L298N \citep{L298} y permite controlar dos motores de corriente continua de manera simultánea e independiente.  Sus características principales son:

\begin{itemize}
	\item Tensión mínima: 5 V.
	\item Tensión máxima: 35 V.
	\item Corriente máxima: 2 A.
	\item Tensión de nivel lógico: 5 V.
	\item Potencia máxima 25 W
	\item Medidas: 43 x 43 x 24 mm
\end{itemize}

La placa tiene la opción de habilitar o no el regulador LM7805 integrado para alimentar la parte lógica. En la figura \ref{fig:Esquema} se observa el diagrama esquemático del módulo.


\begin{figure}[h]
	\centering
	\includegraphics[width=12cm]{./Figures/Modulo.png}
	\caption{diagrama esquemático del módulo.\protect\footnotemark.}
	\label{fig:Esquema}
\end{figure}
\footnotetext{Imagen tomada de \url{http://robots-argentina.com.ar}}
\pagebreak



\subsection{Módulo sensor de Infrarrojos}

Se utilizaron dos módulos sensores de proximidad por infrarrojos IR FC-51 \citep{IR} para la detección de obstáculos por parte del robot. Estos módulos están compuestos por  un emisor de luz infrarroja (IR)  y un receptor que detecta su reflejo en  las superficies contra las que se enfrenta, de modo que presentan una señal en  presencia de cualquier obstáculo en su parte frontal. Un potenciómetro permite al ajustar el rango de detección. 

El sensor presenta una respuesta estable incluso con luz ambiente o en completa oscuridad. En la figura \ref{fig:moduloIR} se observa una imagen del sensor de Infrarrojos.

\begin{figure}[h]
	\centering
	\includegraphics[width=8cm]{./Figures/moduloIR.jpg}
	\caption{Módulo sensor de Infrarrojos .\protect\footnotemark.}
	\label{fig:moduloIR}
\end{figure}
\footnotetext{Imagen tomada de \url{http://robots-argentina.com.ar}}



En la figura \ref{fig:IRschem} se muestra el circuito esquemático  del sensor de Infrarrojos

\begin{figure}[h]
	\centering
	\includegraphics[width=14cm]{./Figures/IRschem.jpg}
	\caption{Esquema del módulo sensor de Infrarrojos .\protect\footnotemark.}
	\label{fig:IRschem}
\end{figure}
\footnotetext{Imagen tomada de \url{http://robots-argentina.com.ar}}
\pagebreak

Las características del módulo son:
\begin{itemize}
	\item Ángulo de cobertura: 35°.
	\item Tensión de funcionamiento: 3 V – 6 V.
	\item Rango de detección: 2 cm – 30 cm (ajustable con el potenciómetro).
	\item Tamaño: 4,5 cm x 1,4 cm x 0.7cm. 
	\item Discriminación: Las salida toma nivel lógico bajo cuando se detecta un obstáculo (reflexión).
\end{itemize}



\subsection{Baterías}

En función del consumo y la intensión de no dedicar mayor espacio a las celdas de alimentación, se emplearon dos baterías de Ion-litio tipo 18650. Una de las ventajas de las Ion-litio es que permiten ser recargadas con una media de entre 600 a 1000 veces sin que se estropeen ni pierdan efectividad \citep{18650}. La capacidad de estas baterías varían de un modelo a otro pero suelen estar comprendidas entre los 2100 maH y los 4000maH.Su tensión nominal es de 3,7 V (hasta 4,2 V en vacío).

Las baterías van conectadas en serie para lograr una tensión de 7,4 V acorde a la alimentación de los motores y con un margen superior necesario para el correcto funcionamiento del regulador de tensión de 5 V del módulo de accionamiento de motores. 
Las baterías se insertaron en un portapilas comercial. En la figura \ref{fig:portapila} se muestra las dos baterías 18650 ya instaladas en su portapila.


\begin{figure}[h]
	\centering
	\includegraphics[width=10cm]{./Figures/portapilas.PNG}
	\caption{Las dos baterías 18650 en su portapila .}
	\label{fig:portapila}
\end{figure}


\subsection{Módulo de comunicaciones Bluetooth}

Se utilizó el módulo Bluetooth HC-05 para la comunicación comunicaciones con el robot \citep{HC05}. El mismo ya había sido utilizado en  prácticas de la asignatura “Protocolos de Comunicaciones en sistemas embebidos”, conectado al puerto serie de la EDU-CIAA. Todos los parámetros del módulo se pueden configurar mediante comandos AT. 

En la figura \ref{fig:moduloHC05} se observa el módulo HC-05.


\begin{figure}[h]
	\centering
	\includegraphics[width=8cm]{./Figures/HC05.jpeg}
	\caption{Módulo Bluetooth HC-05 .\protect\footnotemark.}
	\label{fig:moduloHC05}
\end{figure}
\footnotetext{Imagen tomada de \url{https://maker.pro/custom/tutorial/hc-05-bluetooth-transceiver-module-datasheet-highlights}}



Las características del módulo son:
\begin{itemize}
	\item Voltaje de operación: 3.6 V - 6 V DC.
	\item Consumo corriente: 50 mA.
	\item Bluetooth: V2.0+EDR.
	\item Frecuencia: Banda ISM 2.4 GHz.
	\item Modulación: GFSK(Gaussian Frequency Shift Keying)
	\item Potencia de transmisión: 4 dBm, Class 2.
	\item Sensibilidad: -84dBm a 0.1% BER.
	\item Alcance 10 metros.
	\item Tamaño: 3,7 cm x 1,6 cmm.
\end{itemize}




%----------------------------------------------------------------------------------------
%	SECTION 2
%-----------------------------------------------------------------
\section{Requerimientos}

En esta sección se enumeran los requerimientos planteados en la planificación de trabajo final,elaborada al inicio del proyecto. Se encuentra dividida en requerimientos funcionales y requerimientos no funcionales.

\label{sec:requerimientos}

\begin{enumerate}
\item Requerimientos funcionales
	\begin{enumerate}
	\item Capacidad de locomoción.  El robot debe ser capaz de desplazarse por medio de ruedas motorizadas, a través de superficies planas.
	\item Capacidad de percepción. El robot debe ser capaz de detectar y obtener información del medio. 
	\item Capacidad de comunicación inalámbrica.
	\item El robot deberá funcionar con alimentación a batería recargable.
	\item El proyecto debe ser extensible a una posible herramienta de enseñanza e investigación

	\end{enumerate}
\item Requerimientos no funcionales
	\begin{enumerate}
	\item El robot no debe resultar peligroso para el ambiente o las personas con las que podría interactuar.
	\item El diseño del robot debe respetar regulaciones en cuanto a radiación en el espectro ultravioleta.
	\item Se utilizarán componentes electrónicos disponibles comercialmente en Argentina.
	\end{enumerate}
\end{enumerate}

\section{Planificación}

El trabajo se organizó para ser terminado en el mes de junio de 2021 con una dedicación aproximada de 600 horas en total. Con el fin de organizar y dar seguimiento a las actividades requeridas y poder identificar los desvíos en los tiempos de ejecución programados, se cuantificaron los tiempos de las diversas tareas mediante el diagrama de Gantt, que se observa en las figuras \ref{fig:gantt1} y \ref{fig:gantt2}


\begin{figure}[htpb]
\centering 
\includegraphics[width=\textwidth]{./Figures/gantttabla.PNG}
\caption{Tabla de tareas de Gantt}
\label{fig:gantt1}
\end{figure}

\begin{figure}[htpb]
\centering 
\includegraphics[width=.8\textwidth]{./Figures/Gantt.PNG}
\caption{Diagrama de Gantt}
\label{fig:gantt2}
\end{figure}

\pagebreak

Se confeccionó también un diagrama de Activity on Node, con la finalidad de resaltar las tareas cuyos retrasos podrían resultar críticos para la concreción del trabajo. En rojo se indica el camino crítico, como puede apreciarse en la figura \ref{fig:AoN}

\begin{figure}[htpb]
\centering 
\includegraphics[width=.8\textwidth]{./Figures/AoN.png}
\caption{Diagrama en \textit{Activity on Node}}
\label{fig:AoN}
\end{figure}


A partir de este análisis está organizado el trabajo que se presenta en los próximos capítulos.

 
	\chapter{Diseño e implementación} % Main chapter title

\label{Chapter3} % Change X to a consecutive number; for referencing this chapter elsewhere, use \ref{ChapterX}

\definecolor{mygreen}{rgb}{0,0.6,0}
\definecolor{mygray}{rgb}{0.5,0.5,0.5}
\definecolor{mymauve}{rgb}{0.58,0,0.82}

%%%%%%%%%%%%%%%%%%%%%%%%%%%%%%%%%%%%%%%%%%%%%%%%%%%%%%%%%%%%%%%%%%%%%%%%%%%%%
% parámetros para configurar el formato del código en los entornos lstlisting
%%%%%%%%%%%%%%%%%%%%%%%%%%%%%%%%%%%%%%%%%%%%%%%%%%%%%%%%%%%%%%%%%%%%%%%%%%%%%
\lstset{ %
  backgroundcolor=\color{white},   % choose the background color; you must add \usepackage{color} or \usepackage{xcolor}
  basicstyle=\footnotesize,        % the size of the fonts that are used for the code
  breakatwhitespace=false,         % sets if automatic breaks should only happen at whitespace
  breaklines=true,                 % sets automatic line breaking
  captionpos=b,                    % sets the caption-position to bottom
  commentstyle=\color{mygreen},    % comment style
  deletekeywords={...},            % if you want to delete keywords from the given language
  %escapeinside={\%*}{*)},          % if you want to add LaTeX within your code
  %extendedchars=true,              % lets you use non-ASCII characters; for 8-bits encodings only, does not work with UTF-8
  %frame=single,	                % adds a frame around the code
  keepspaces=true,                 % keeps spaces in text, useful for keeping indentation of code (possibly needs columns=flexible)
  keywordstyle=\color{blue},       % keyword style
  language=[ANSI]C,                % the language of the code
  %otherkeywords={*,...},           % if you want to add more keywords to the set
  numbers=left,                    % where to put the line-numbers; possible values are (none, left, right)
  numbersep=5pt,                   % how far the line-numbers are from the code
  numberstyle=\tiny\color{mygray}, % the style that is used for the line-numbers
  rulecolor=\color{black},         % if not set, the frame-color may be changed on line-breaks within not-black text (e.g. comments (green here))
  showspaces=false,                % show spaces everywhere adding particular underscores; it overrides 'showstringspaces'
  showstringspaces=false,          % underline spaces within strings only
  showtabs=false,                  % show tabs within strings adding particular underscores
  stepnumber=1,                    % the step between two line-numbers. If it's 1, each line will be numbered
  stringstyle=\color{mymauve},     % string literal style
  tabsize=2,	                   % sets default tabsize to 2 spaces
  title=\lstname,                  % show the filename of files included with \lstinputlisting; also try caption instead of title
  morecomment=[s]{/*}{*/}
}
En este capítulo se enumeran y desarrollan los aspectos considerados a la hora diseñar el robot. Se tuvieron en cuenta los alcances establecidos como así también las posibilidades económicas de solventar el proyecto.
%----------------------------------------------------------------------------------------
%	SECTION 1
%----------------------------------------------------------------------------------------
\section{Diseño de hardware}

En esta sección se detallan los componentes y módulos electrónicos que forman parte del robot. Se detalla la función que desempeña cada uno de ellos. En la figura \ref{fig:diagramaini} se puede apreciar un diagrama en bloques de los módulos que conforman el robot.


\begin{figure}[h]
	\centering
	\includegraphics[width=12cm]{./Figures/diagini.PNG}
	\caption{Diagrama en bloques del robot.}
	\label{fig:diagramaini}
\end{figure}



	\subsection{Poncho}
Se denominación  “Poncho”  se utiliza entre la comunidad del proyecto CIAA para referirse a una  placa de expansión de “shield”  que se conecta sobre algún procesador de la familia CIAA.  Para este proyecto se diseñó un poncho para facilitar las conexiones de la placa EDU-CIAA con los sensores, actuadores y el módulo de comunicación bluetooth.

		\subsubsection{Diseño esquemático del poncho}

El poncho consta de conectores para las señales de entrada:

\begin{itemize}
	\item Sensores infrarrojos (2).
	\item Finales de carrera (2).
	\item Sensor de movimiento (1).
\end{itemize}

A su vez, permite la conexión con los dispositivos de salida

\begin{itemize}
	\item Módulo de control de motores.
	\item Relé actuador (en la placa).
	\item Buzzer y LEDs (en la placa).
\end{itemize}

La placa posee conexionado para montar un módulo HC-05 de comunicación bluetooth y un conector destinado a dispositivos I2C (como podría ser un módulo de giróscopo o acelerómetro).

		\subsubsection{Diseño PCB del poncho}


La placa fue diseñada durante la cursada de la asignatura “diseño de circuitos impresos", según los lineamientos expuestos en la documentación para ponchos CIAA. Se procedió a la fabricación de la placa utilizando medios caseros de manufactura.

El diseño de PCB se realizó con el software KiCad \citep{KiCad} (Versión 5.1.9), el cual es un paquete de software para el diseño de circuitos electrónicos o EDA (Electronic Design
Automation). En la figura \ref{fig:poncho3d} se observa el modelo 3D de la placa y sus componentes.


\begin{figure}[h]
	\centering
	\includegraphics[width=11cm]{./Figures/ponchoiso.PNG}
	\caption{Vista del modelo 3D del poncho rUVot.}
	\label{fig:poncho3d}
\end{figure}


En la figura \ref{fig:esquematico} se presenta el circuito esquemático del poncho donde se puede observar el conexionado eléctrico.

%\begin{figure}[h]
%	\centering
%	\includegraphics[width=\textwidth]{./Figures/esquematico.png}
%	\caption{Circuito esquemático del poncho.}
%	\label{fig:esquematico}
%\end{figure}
%\pagebreak

\subsection{Esquema de  de comunicaciones}

\section{Diseño mecánico}
\subsection{Gabinete del robot}
\subsection{Motores}

\section{Diseño de software}
\subsection{Tarea de control de motores}
\subsection{Tarea de comunicaciones}







%
%
%
%
%\begin{verbatim}
%\begin{lstlisting}[caption= "un epígrafe descriptivo"]
%	las líneas de código irían aquí...
%\end{lstlisting}
%\end{verbatim}
%
%A modo de ejemplo:
%
%\begin{lstlisting}[label=cod:vControl,caption=Pseudocódigo del lazo principal de control.]  % Start your code-block
%
%#define MAX_SENSOR_NUMBER 3
%#define MAX_ALARM_NUMBER  6
%#define MAX_ACTUATOR_NUMBER 6
%
%uint32_t sensorValue[MAX_SENSOR_NUMBER];		
%FunctionalState alarmControl[MAX_ALARM_NUMBER];	//ENABLE or DISABLE
%state_t alarmState[MAX_ALARM_NUMBER];						//ON or OFF
%state_t actuatorState[MAX_ACTUATOR_NUMBER];			//ON or OFF
%
%void vControl() {
%
%	initGlobalVariables();
%	
%	period = 500 ms;
%		
%	while(1) {
%
%		ticks = xTaskGetTickCount();
%		
%		updateSensors();
%		
%		updateAlarms();
%		
%		controlActuators();
%		
%		vTaskDelayUntil(&ticks, period);
%	}
%}
%\end{lstlisting}




	% Chapter Template

\chapter{Ensayos y resultados} % Main chapter title

\label{Chapter4} % Change X to a consecutive number; for referencing this chapter elsewhere, use \ref{ChapterX}

%----------------------------------------------------------------------------------------
%	SECTION 1
%----------------------------------------------------------------------------------------

\section{Pruebas funcionales del hardware}
\label{sec:pruebasHW}

En este capítulo se detallan los ensayos realizados para comprobar el correcto funcionamiento de hardware y firmware, y la interacción de los módulos que componen el robot.

\subsection{Validación de movimientos del robot}
\subsection{Validación módulo de comunicaciones Bluetooth}
\subsection{Validación detección de obstáculos}
\subsection{Validación de navegación autónoma}


\section{Pruebas no Funcionales}
%\subsection{Tarea de comunicaciones} 
	% Chapter Template

\chapter{Conclusiones} % Main chapter title

\label{Chapter5} % Change X to a consecutive number; for referencing this chapter elsewhere, use \ref{ChapterX}


%----------------------------------------------------------------------------------------

%----------------------------------------------------------------------------------------
%	SECTION 1
%----------------------------------------------------------------------------------------


En este capítulo se presenta un breve resumen del trabajo realizado, los problemas encontrados y los resultados obtenidos. También se mencionan mejoras a realizar en el futuro.

\section{Resultados obtenidos}

El trabajo finalizó con el desarrollo de un prototipo de robot móvil para tareas de desinfección por efecto de rayos ultravioletas germicidas. Se cumplieron los requerimientos planteados en la planificación del trabajo.
Se desarrolló con éxito un circuito impreso como placa de expansión de hardware, y un firmware funcional para la placa EDU-CIAA  
Se verificó el funcionamiento en el modo autónomo, en el que realiza un recorrido evitando obstáculos, como así también  en el  modo de teleoperación, en el que puede controlarse a distancia desde una  aplicación en un celular o Tablet.
El dispositivo puede ser usado para desinfección sin residuos químicos en espacios públicos y en el hogar. 

La planificación, se cumplió dentro de los plazos esperados, aunque se manifestó el riesgo “Imposibilidad de cumplir con los plazos planteados para el desarrollo del proyecto”. Esto se debió a la  reducción de tiempo disponible para dedicarlo al proyecto,  debido a actividades laborales y estar cursando las últimas materias de le carrera.  Al haber extendido el plazo para la entrega y haber re-planificado actividades se logró mitigar este inconveniente.


%----------------------------------------------------------------------------------------
%	SECTION 2
%----------------------------------------------------------------------------------------
\section{Conocimientos aplicados}

Durante la realización de este trabajo se aplicaron conocimientos adquiridos en el transcurso de la carrera de especialización. 
En particular, fueron importantes los aportes de las siguiente  asignaturas:


\begin{itemize}
	\item Gestión de proyectos, para realizar la planificación y generar toda la documentación inicial.
	\item Ingeniería de software para definir los requerimientos básicos y pensar el proyecto desde las necesidades del usuario.También se aplicaron los conocimientos relativos a la implementación de un repositorio GIT para el resguardo y versionado de toda la documentación del proyecto.
	\item Programación de microcontroladores para la implementación del firmware en C del microcontrolador ARM Cortex-M4 de la placa EDU-CIAA. En la asignatura se presentó todo lo referente a la modularización por archivos implementada en este trabajo y el modelo de máquinas de estado finito.
	\item Protocolos de comunicaciones en sistemas embebidos, para conocer las posibilidades de comunicación de la placa EDU-CIAA con otros dispositivos, en particular con el módulo Bluetooth. 
	\item Diseño de Circuitos Impresos, para el desarrollo de la placa de expansión de hardware (poncho) utilizada en este trabajo, y el aprendizaje de buenas costumbres de diseño de PCB
			
\end{itemize}

%----------------------------------------------------------------------------------------
%	SECTION 3
%----------------------------------------------------------------------------------------
\section{Próximos pasos}

Como mejoras a futuro se contempla:

\begin{itemize}
	\item Agregar una unidad de medición inercial o IMU (por su sigla en inglés) como ser un acelerómetro o un giróscopo, para tener informa acerca de la velocidad y orientación del robot  en el modo autónomo. De esta manera se podría ampliar la variedad de recorridos posibles y que no dependan únicamente de las características del entorno. 
	\item Al contar con un puerto I2C en la placa, sería posible incorporar un lector de tarjetas de memoria (tipo SD) para almacenar allí la librería con la que se configura la máquina de estados principal. Con este aditamento sería posible definir o ampliar el comportamiento autónomo del robot sin necesidad de modificar su programación. 		
	\item Ya que la placa de expansión de hardware utiliza un relé para conmutar el módulo UVC, podían desarrollarse otros módulos (intercambiables) con su propia alimentación, que utilicen diferentes lámparas germicidas o que ofrezcan otras prestaciones.  
\end{itemize} 
\end{verbatim}

Los apéndices también deben escribirse en archivos .tex separados, que se deben ubicar dentro de la carpeta \emph{Appendices}. Los apéndices vienen comentados por defecto con el caracter \code{\%} y para incluirlos simplemente se debe eliminar dicho caracter.

Finalmente, se encuentra el código para incluir la bibliografía en el documento final.  Este código tampoco debe modificarse. La metodología para trabajar las referencias bibliográficas se desarrolla en la sección \ref{sec:biblio}.
%----------------------------------------------------------------------------------------

\section{Bibliografía}
\label{sec:biblio}

Las opciones de formato de la bibliografía se controlan a través del paquete de latex \option{biblatex} que se incluye en la memoria en el archivo memoria.tex.  Estas opciones determinan cómo se generan las citas bibliográficas en el cuerpo del documento y cómo se genera la bibliografía al final de la memoria.

En el preámbulo se puede encontrar el código que incluye el paquete biblatex, que no requiere ninguna modificación del usuario de la plantilla, y que contiene las siguientes opciones:

\begin{lstlisting}
\usepackage[backend=bibtex,
	natbib=true, 
	style=numeric, 
	sorting=none]
{biblatex}
\end{lstlisting}

En el archivo \file{reference.bib} se encuentran las referencias bibliográficas que se pueden citar en el documento.  Para incorporar una nueva cita al documento lo primero es agregarla en este archivo con todos los campos necesario.  Todas las entradas bibliográficas comienzan con $@$ y una palabra que define el formato de la entrada.  Para cada formato existen campos obligatorios que deben completarse. No importa el orden en que las entradas estén definidas en el archivo .bib.  Tampoco es importante el orden en que estén definidos los campos de una entrada bibliográfica. A continuación se muestran algunos ejemplos:

\begin{lstlisting}
@ARTICLE{ARTICLE:1,
    AUTHOR="John Doe",
    TITLE="Title",
    JOURNAL="Journal",
    YEAR="2017",
}
\end{lstlisting}


\begin{lstlisting}
@BOOK{BOOK:1,
    AUTHOR="John Doe",
    TITLE="The Book without Title",
    PUBLISHER="Dummy Publisher",
    YEAR="2100",
}
\end{lstlisting}


\begin{lstlisting}
@INBOOK{BOOK:2,
    AUTHOR="John Doe",
    TITLE="The Book without Title",
    PUBLISHER="Dummy Publisher",
    YEAR="2100",
    PAGES="100-200",
}
\end{lstlisting}


\begin{lstlisting}
@MISC{WEBSITE:1,
    HOWPUBLISHED = "\url{http://example.com}",
    AUTHOR = "Intel",
    TITLE = "Example Website",
    MONTH = "12",
    YEAR = "1988",
    URLDATE = {2012-11-26}
}
\end{lstlisting}

Se debe notar que los nombres \emph{ARTICLE:1}, \emph{BOOK:1}, \emph{BOOK:2} y \emph{WEBSITE:1} son nombres de fantasía que le sirve al autor del documento para identificar la entrada. En este sentido, se podrían reemplazar por cualquier otro nombre.  Tampoco es necesario poner : seguido de un número, en los ejemplos sólo se incluye como un posible estilo para identificar las entradas.

La entradas se citan en el documento con el comando: 

\begin{verbatim}
\citep{nombre_de_la_entrada}
\end{verbatim}

Y cuando se usan, se muestran así: \citep{ARTICLE:1}, \citep{BOOK:1}, \citep{BOOK:2}, \citep{WEBSITE:1}.  Notar cómo se conforma la sección Bibliografía al final del documento. 
